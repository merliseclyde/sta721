\documentclass{beamer}
%\documentclass[handout]{beamer}
%\usepackage[dvips]{color}
\usepackage{graphicx}
\usepackage{amsmath,amssymb,array,comment,eucal}
\newcommand{\e}{\mathbf{e}}
\renewcommand{\P}{\mathbf{P}}
\newcommand{\F}{\mathbf{F}}
\newcommand{\R}{\textsf{R}}
\newcommand{\mat}[1] {\mathbf{#1}}
%\newcommand{\ind}{\mathrel{\mathop{\sim}\limits^{\mathit{ind}}}}
%\newcommand{\iid}{\mathrel{\mathop{\sim}\limits^{\mathit{iid}}}}
\newcommand{\E}{\textsf{E}}
\newcommand{\SE}{\textsf{SE}}
\newcommand{\SSE}{\textsf{SSE}}
\newcommand{\RSS}{\textsf{RSS}}
\newcommand{\FSS}{\textsf{FSS}}
\renewcommand{\SS}{\textsf{SS}}
\newcommand{\MSE}{\textsf{MSE}}
\newcommand{\SSR}{\textsf{SSR}}
\newcommand{\Be}{\textsf{Beta}}
\newcommand{\St}{\textsf{St}}
%\newcommand{\C}{\textsf{C}}
\newcommand{\GDP}{\textsf{GDP}}
\newcommand{\NcSt}{\textsf{NcSt}}
\newcommand{\Bin}{\textsf{Bin}}
\newcommand{\NB}{\textsf{NegBin}}
\renewcommand{\NG}{\textsf{NG}}
\newcommand{\N}{\textsf{N}}
\newcommand{\Ber}{\textsf{Ber}}
\newcommand{\Poi}{\text{Poi}}
\newcommand{\Gam}{\textsf{Gamma}}
\newcommand{\BB}{\textsf{BB}}
\newcommand{\Gm}{\textsf{G}}
\newcommand{\Un}{\textsf{Unif}}
\newcommand{\Ex}{\textsf{Exp}}
\newcommand{\DE}{\textsf{DE}}
\newcommand{\tr}{\textsf{tr}}
\newcommand{\cF}{{\cal{F}}}
\newcommand{\cL}{{\cal{L}}}
\newcommand{\cI}{{\cal{I}}}
\newcommand{\cB}{{\cal{B}}}
\newcommand{\cP}{{\cal{P}}}
\newcommand{\bbR}{\mathbb{R}}
\newcommand{\bbN}{\mathbb{N}}
\newcommand{\pperp}{\mathrel{{\rlap{$\,\perp$}\perp\,\,}}}
\newcommand{\OFP}{(\Omega,\cF, \P)}
\newcommand{\eps}{\boldsymbol{\epsilon}}
\newcommand{\1}{\mathbf{1}_n}
\newcommand{\gap}{\vspace{8mm}}
\newcommand{\ind}{\mathrel{\mathop{\sim}\limits^{\rm ind}}}
\newcommand{\simiid}{\ensuremath{\mathrel{\mathop{\sim}\limits^{\rm
iid}}}}
\newcommand{\eqindis}{\ensuremath{\mathrel{\mathop{=}\limits^{\rm D}}}}
\newcommand{\iid}{\textit{i.i.d.}}
\newcommand{\SSZ}{S_{zz}}
\newcommand{\SZW}{S_{zw}}
\newcommand{\Var}{\textsf{Var}}
\newcommand{\corr}{\textsf{corr}}
\newcommand{\diag}{\textsf{diag}}
\newcommand{\var}{\textsf{var}}
\newcommand{\Cov}{\textsf{Cov}}
\newcommand{\Sam}{{\cal S}}
\def\H{\mathbf{H}}
\newcommand{\I}{\mathbf{I}}
\newcommand{\Y}{\mathbf{Y}}
\newcommand{\tY}{\tilde{\mathbf{Y}}}
\newcommand{\Yhat}{\hat{\mathbf{Y}}}
\newcommand{\Yobs}{\mathbf{Y}_{{\cal S}}}
\newcommand{\barYobs}{\bar{Y}_{{\cal S}}}
\newcommand{\barYmiss}{\bar{Y}_{{\cal S}^c}}
\def\bv{\mathbf{b}}
\def\X{\mathbf{X}}
\def\tX{\tilde{\mathbf{X}}}
\def\x{\mathbf{x}}
\def\xbar{\bar{\mathbf{x}}}
\def\Xbar{\bar{\mathbf{X}}}
\def\Xg{\mathbf{X}_{\boldsymbol{\gamma}}}
\def\Ybar{\bar{\Y}}
\def\ybar{\bar{y}}
\def\y{\mathbf{y}}
\def\Yf{\mathbf{Y_f}}
\def\W{\mathbf{W}}
\def\L{\mathbf{L}}
\def\w{\mathbf{w}}
\def\U{\mathbf{U}}
\def\V{\mathbf{V}}
\def\Q{\mathbf{Q}}
\def\Z{\mathbf{Z}}
\def\z{\mathbf{z}}
\def\v{\mathbf{v}}
\def\u{\mathbf{u}}

\def\zero{\mathbf{0}}
\def\one{\mathbf{1}}
\newcommand{\taub}{\boldsymbol{\tau}}
\newcommand{\betav}{\boldsymbol{\beta}}
\newcommand{\alphav}{\boldsymbol{\alpha}}
\newcommand{\A}{\mathbf{A}}
\def\a{\mathbf{a}}
\def\K{\mathbf{K}}
\newcommand{\B}{\mathbf{B}}
\def\b{\boldsymbol{\beta}}
\def\bhat{\hat{\boldsymbol{\beta}}}
\def\btilde{\tilde{\boldsymbol{\beta}}}
\def\tb{\tilde{\boldsymbol{\beta}}}
\def\bg{\boldsymbol{\beta_\gamma}}
\def\bgnot{\boldsymbol{\beta_{(-\gamma)}}}
\def\mub{\boldsymbol{\mu}}
\def\tmub{\tilde{\boldsymbol{\mu}}}
\def\muhat{\hat{\boldsymbol{\mu}}}
\def\t{\boldsymbol{\theta}}
\def\tk{\boldsymbol{\theta}_k}
\def\tj{\boldsymbol{\theta}_j}
\def\Mk{\boldsymbol{{\cal M}}_k}
\def\M{\boldsymbol{{\cal M}}}
\def\Mj{\boldsymbol{{\cal M}}_j}
\def\Mi{\boldsymbol{{\cal M}}_i}
\def\Mg{{\boldsymbol{{\cal M}_\gamma}}}
\def\Mnull{\boldsymbol{{\cal M}}_{N}}
\def\gMPM{\boldsymbol{\gamma}_{\text{MPM}}}
\def\gHPM{\boldsymbol{\gamma}_{\text{HPM}}}
\def\Mfull{\boldsymbol{{\cal M}}_{F}}
\def\tg{\boldsymbol{\theta}_{\boldsymbol{\gamma}}}
\def\g{\boldsymbol{\gamma}}
\def\eg{\boldsymbol{\eta}_{\boldsymbol{\gamma}}}
\def\G{\mathbf{G}}
\def\cM{\cal M}
\def\D{\Delta}
\def \shat{{\hat{\sigma}}^2}
\def\uv{\mathbf{u}}
\def\l {\lambda}
\def\d{\delta}
\def\Sigmab{\boldsymbol{\Sigma}}
\def\Lambdab{\boldsymbol{\Lambda}}
\def\lambdab{\boldsymbol{\lambda}}
\def\Mg{{\cal M}_\gamma}
\def\S{{\cal{S}}}
\def\qg{p_{\boldsymbol{\gamma}}}
\def\pg{p_{\boldsymbol{\gamma}}}
\def\t{\boldsymbol{\theta}}  
\def\T{\boldsymbol{\Theta}}  
\usepackage{verbatim}

\usetheme{Warsaw}
\usecolortheme{orchid}
\title{Models \& Estimation}
\subtitle{Merlise Clyde}
\author{STA721 Linear Models}
\institute{Duke University}
\date{August 28, 2014}
\logo{duke.eps}

\begin{document}
\maketitle

\begin{frame}\frametitle{Outline}
  \tableofcontents

Readings: Christensen Chapter 1-2, Appendix A
\end{frame}
%\section{Models}
\begin{frame}
  \frametitle{Models}
Take an random vector $\Y \in \bbR^n$ which is observable and decompose  \pause
$$ \Y = \alert<2>{\mub} + \alert<3>{\eps}$$
into  \alert<2>{$\mub \in \bbR^n$ (unknown, fixed)} \pause and
\alert<3>{$\eps \in \bbR^n$ unobservable error vector (random)}
\pause

\vspace{18pt}
Usual assumptions? \pause
\begin{itemize}
\item $\E[\eps] = \zero$ \pause  $ \quad \Rightarrow \E[\Y] = \mub$
  (mean vector) \pause
\item $\Cov[\eps] = \sigma^2 \I_n$   \pause  $ \quad \Rightarrow
  \Cov[\Y] = \sigma^2 \I_n$  (errors are uncorrelated) \pause
\item $\eps \sim \N(\zero, \sigma^2 \I)$ \pause  $ \quad \Rightarrow
  \Y \sim \N(\mub, \sigma^2 \I)$ \pause
\end{itemize}

The distribution assumption allows us to right down a likelihood function
\end{frame}

\begin{frame} \frametitle{Likelihood Functions}

The likelihood function for $\mub, \sigma^2$ is proportional to the
sampling distribution of the data \pause

\begin{eqnarray*}
 \cL(\mub, \sigma^2) & \propto & (\alert<3>{2 \pi} \sigma^2)^{-n/2} 
 \exp{\left\{ - \frac 1 2 \frac{(\Y - \mub)^T (\Y - \mub)}{\sigma^2} 
\right\}}  \pause \\
   & \propto & (\sigma^2)^{-n/2} 
 \exp{\left\{ - \frac 1 2 \frac{\| \Y - \mub \|^2}{\sigma^2} 
\right\}}  \pause
\end{eqnarray*}
Find values of $\muhat$ and $\shat$ that maximize the likelihood
$\cL(\mub, \sigma^2)$ for $\mub \in \bbR^n$ and $\sigma^2 \in \bbR^+$
\pause

\vspace{18pt}
Clearly, $\muhat = \Y$ but $\shat = 0$  is outside the parameter space
\end{frame}
%\subsection{Oneway Anova Model}
\begin{frame}
  \frametitle{Restrictions on $\mub$}
Linear model $\mub = \X \b$ for fixed $\X$ ($n \times p$) and some $\b
\in \bbR^p$  restrict the possible values for $\mub$  \pause

\begin{itemize}
\item  Oneway Anova Model  $j = 1, \ldots, J$ groups with $n_j$
  replicates in each group $j$ \pause
\item {``cell  means'' $Y_{ij} = \mu_j + \epsilon_{ij}$ }


$$\mub = {\left[
  \begin{array}{llll}
    \one_{n_1} & \zero_{n_1} & \ldots & \zero_{n_1} \\
    \zero_{n_2} & \one_{n_2} & \ldots & \zero_{n_2} \\
    \vdots & \vdots & \ddots & \vdots \\
    \zero_{n_J} & \ldots & \zero_{n_J} & \one_{n_J} \\
  \end{array}
   \right] \left(
     \begin{array}{l}
\mu_1 \\ \mu_2 \\ \vdots \\ \mu_J 
    \end{array}
\right) }$$
\item $\one_{n_j}$ is a vector of length $n_j$ of ones. 
\item  $\zero_{n_j}$ is a vector of length $n_jh$ of zeros.
\end{itemize}

\end{frame}
\begin{frame}
  \frametitle{Equivalent Model}
Linear model $\mub = \X \b$ for fixed $\X$ ($n \times p$) and some $\b
\in \bbR^p$    \pause

\begin{itemize}
\item  Oneway Anova Model  $j = 1, \ldots, J$ groups with $n_j$
  replicates in each group $j$ \pause
\item 
{``Treatment effects'' $Y_{ij} = \mu + \tau_j + \epsilon_{ij}$}
 $$ \mub = {\left[
  \begin{array}{lllll}
  \one_{n_1} &  \one_{n_1} & \zero_{n_1} & \ldots & \zero_{n_1} \\
    \one_{n_2} &  \zero_{n_2} & \one_{n_2} & \ldots & \zero_{n_2} \\
  \vdots &    \vdots & \vdots & \ddots & \vdots \\
  \one_{n_J} & \zero_{n_J} & \ldots & \zero_{n_J} & \one_{n_J} \\
  \end{array}
   \right] \left(
     \begin{array}{l}
\mu \\ \tau_1 \\ \vdots \\ \tau_J 
    \end{array}
\right)}
$$
 \pause
\item Equivalent means $\mu_j = \mu + \tau_j$
\item  Should our inference for $\mub$ depend on how we represent or
  parameterize $\mub$?
\end{itemize}

\end{frame}

\begin{frame}
  \frametitle{Column Space}
Many equivalent ways to represent the same mean vector -- inference
should be independent of the coordinate system used \pause
\begin{itemize}
\item Let $\X_1, \X_2,\ldots, \X_p \in \bbR^n$  \pause
\item The set of all linear combinations of $\X_1, \ldots, \X_p$ is
  the space spanned by $\X_1, \ldots, \X_p \equiv S(\X_1, \ldots,
  \X_p)$  \pause
\item Let $\X = [ \X_1\X_2 \ldots \X_p]$ be a $n \times p$ matrix with
  columns $\X_j$ then the column space of $\X$, $\alert<4>{C(\X) = S(\X_1, \ldots,
  \X_p)}$ space spanned by the (column) vectors of $\X$ \pause
\item $\mub \in C(\X): C(\X) = \{\mub \mid \mub \in \bbR^n \text{ such
    that } \X \b =
    \mub $ for some $\b \in \bbR^p\}$  (also called the Range of $\X$,
    $R(\X)$ \pause
\item $\b$ are the ``coordinates'' of $\mub$  in this space \pause
\item $C(\X)$ is a subspace of $\bbR^n$
\end{itemize}
\end{frame}
%\section{Vector Spaces \& Subspaces}
\begin{frame}
  \frametitle{Vector spaces}
A collection of vectors $V$ is a real {\bf vector
  space} if the following conditions hold: for any pair $\x$ and $\y$
of vectors in $V$ there corresponds a vector $\x + \y$ and scalars
$\alpha, \beta \in \bbR$ such that: \pause
\begin{small}
\begin{enumerate}
\item $\x + \y = \y + \x$  (vector addition is commutative) \pause
\item $(\x + \y) + \z = \x + (\y + \z)$ (vector addition is
  associative) \pause
\item there exists a unique vector $\zero \in V$ (the origin) such that $\x + \zero =
 \x$ for every vector $\x$ \pause
\item for every $\x \in V$, there exists a unique vector $-\x$ such that $\x + 
(-\x) = \zero$ \pause
\item $\alpha (\beta \x) = (\alpha \beta) \x $ (multiplication by
  scalars is associative) \pause
\item $1 \x = \x$ for every $\x$ \pause 
\item $\alpha(\x + \y) = \alpha \x + \alpha \y$ (multiplication by
  scalars is distributive with respect vector addition) \pause
\item $(\alpha + \beta)\x = \alpha \x + \beta \x$ (multiplication by scalars is 
distributive with respect to vector addition)
\end{enumerate}
  
\end{small}

\end{frame}
\begin{frame}
  \frametitle{Subspaces}
  \begin{block}{Definition}
    Let $V$ be a vector space and let $V_0$ be a set with $V_0
    \subseteq V$.  $V_0$ is a {\it subspace} of $V$ if and only if
    $V_0$ is a vector space.
  \end{block} \pause

  \begin{theorem}
    Let $V$ be a vector space, and let $V_0$ be a non-empty subset of
    $V$.  If $V_0$ is closed on vector addition and scalar
    multiplication, then $V_0$ is a subspace of $V$
  \end{theorem}  \pause

  \begin{theorem}
    Let $V$ be a vector space, and let $\x_1, \ldots, \x_r$ be vectors
    in    $V$.  The set of all linear combinations of $\x_1, \ldots, \x_r$
    is a subspace of $V$.  \pause
  \end{theorem}
The Column space of $\X$, $C(\X)$, is a subspace of $\bbR^n$
\end{frame}
%\subsection{Basis}
\begin{frame}
  \frametitle{Basis}
  \begin{definition}
    A finite set of vectors $\{\x_i\}$ is {\it linearly dependent }
      if there exist scalars $\{a_i \}$ (not all zero) such that 
 $\sum_i    a_i \x_i = \zero$.
\pause  Conversely  $\sum_i
        a_i \x_i = \zero$ implies that $a_i = 0 \, \forall \,  i$ the set is
        {\it linearly independent}.
  \end{definition} \pause

  \begin{block}{Definition}
    If $V_0$ is a subspace of $V$ and if $\{\x_1, \ldots, \x_r \}$ is a
    linearly independent spanning set for $V_0$, then $\{\x_1, \ldots,
    \x_r \}$  is a basis for $V_0$
  \end{block} \pause

  \begin{theorem}
    If $V_0$ is a subspace of $V$, then all bases for $V_0$ have the
    same number of vectors
  \end{theorem} \pause
Can both the collection of vectors in the cell means 
and the treatment effects parameterizations be a basis?
\end{frame}
\begin{frame} \frametitle{Rank}
  \begin{definition}
    The rank of a subspace $V_0$ is the number of elements in a basis
    for $V_0$ and is written as $r(V_0)$.  Similarly if $\A$ is a
    matrix, the rank of $C(\A)$ is called the rank of $\A$ and is
    written $r(\A)$.
  \end{definition} \pause
What is the rank of the subspace in the Oneway ANOVA model?
\end{frame}
%\subsection{Orthogonality}
\begin{frame}
  \frametitle{Orthogonality}
  \begin{definition}
An inner product space is a vector space $V$ equipped with an inner product: $\langle
\cdot, \cdot \rangle$ is a mapping $V \times V \rightarrow \bbR$.  Two vectors are orthogonal if $\langle \x, \y
    \rangle = 0$, written $\x \perp \y$ \pause   (usual $\x^T\y = 0$)
  \end{definition}
  \begin{definition}
    Two subspaces are orthogonal if $\x \in M$  and $y \in N$ imply
    that $\x \perp \y$ \pause
  \end{definition}
  \begin{definition}
    Orthonormal basis:  $\{\x_1, \ldots, \x_r$\} is an orthonormal
    basis  (ONB) for $M$ if $\langle \x_i, \x_j \rangle = 0$ for $i \neq j$ and
    $\langle \x_i, \x_i \rangle = 1$ for all $i$.  Length $\x = \|
    \x \| = \sqrt{\langle\x, \x \rangle}$.  Distance of two vectors is 
$\| \x - \y \|$
  \end{definition}
\end{frame}
\begin{frame} \frametitle{Oneway Anova}
  Find an orthonormal basis for the oneway ANOVA model.  Is it unique?
  \pause

\vspace{2in}
Can also use Gram-Schmidt sequential orthogonalization
\end{frame}
\begin{frame}
  \frametitle{Decomposition}
  \begin{definition}
    If $M$ and $N$ are subspaces of $V$ that satisfy
    \begin{itemize}
    \item $M \cap N = \{ \zero \}$ \pause
    \item $M + N = V$ where $M+N = \{ \z \mid \z = \x + \y, \  \x \in
      M, \y \in N\}$  ($M \oplus N = V$) Direct Sum \pause
    \end{itemize}
then $M$ and $N$ are complementary spaces. \pause  $N$ is the
complement of $M$ \pause
\end{definition}
\begin{definition}
  If $M$ and $N$ are complementary spaces $M + N =\bbR^n$ and $M$ and
  $N$ are orthogonal subspaces, then $M$ is the orthogonal complement of
  $N$, $N^{\perp}$. 
\end{definition} \pause If $\z\in \bbR^n = N+M$, then we can uniquely
decompose it into a part $\x \in M$ and $\y \in N$ and $r(M) + r(N) =
n$
\end{frame}
%\subsection{Projections}
\begin{frame} \frametitle{Projections}
\begin{definition}
$\P$ is an orthogonal projection operator onto $C(\X)$ if and only if \pause
  \begin{itemize}
  \item $\u \in C(\X)$ implies $\P \u = \u$ (projection) \pause
  \item $\v \perp C(\X)$ implies $\P \v = \zero$ (perpendicular) \pause
  \end{itemize}
\end{definition} 
\begin{itemize}
\item $C(\X)$ and $C(\X)^\perp$ are complementary spaces \pause
\item $\y = \u + \v$ for $\y \in \bbR^n$, $\u \in C(\X)$ and $\v \perp
  C(\X)$  then $\P \y = \P \u + \P \v = \u$ \pause
\item $\u$ is the orthogonal projection of $\y$ onto $C(\X)$

\end{itemize}



\end{frame}

\begin{frame} \frametitle{More on Projections}
\begin{block}{Prop}
   $\P$ is an orthogonal projection operator on $C(\X)$ then $C(\X) = C(\P)$
\end{block} \pause

\begin{theorem}
$\P$ is an orthogonal projection operator on $C(\P)$ if and
only if  \pause
\begin{itemize}
\item $\P = \P^2$  (idempotent)  \pause
\item $\P = \P^T$ (symmetry)   \pause
\end{itemize}
\end{theorem}
Claim:  If $\X$ is $n \times p$ and $r(\X) = p$ then, $\P_{\X} =
\X(\X^T\X)^{-1} \X^T$   operator  onto $C(\X)$
  
\end{frame}
\begin{frame} \frametitle{Projections}
Claim:  If $\X$ is $n \times p$ and $r(\X) = p$ then, $\P_{\X} =
\X(\X^T\X)^{-1} \X^T$  is an orthogonal projection operator  onto
$C(\X)$ \pause
 \begin{itemize}
\item $\P = \P^2$  (idempotent)  \pause
  \begin{eqnarray*}
 \P_X^2 = \P_\X \P_\X & = &\X(\X^T\X)^{-1} \X^T\X(\X^T\X)^{-1} \X^T   \pause \\
             & = & \X(\X^T\X)^{-1}\X^T \pause\\
 & = & \P_{\X} \pause
  \end{eqnarray*}

\item $\P = \P^T$ (symmetry)  \pause
 \begin{eqnarray*}
 \P_\X^T  & = & (\X(\X^T\X)^{-1}\X^T)^T \pause \\
             & = & (\X^T)^T((\X^T\X)^{-1})^T(\X)^T \pause \\
 & = &  \X(\X^T\X)^{-1}\X^T \pause \\
 & = & \P_{\X} \pause
  \end{eqnarray*}
\item $C(X) = C(\P_X)$  
\end{itemize}
\end{frame}
\begin{frame} \frametitle{Projections}
  Claim: $\I - \P_\X$ is an orthogonal projection onto $C(\X)^{\perp}$ \pause
  \begin{itemize}
  \item idempotent  \pause
    \begin{eqnarray*}
(\I - \P_{\X})^2 &= & (\I - \P_\X)(\I - \P_\X)  \pause \\
& = & \I - \P_\X - \P_\X + \P_\X \P_\X       \pause \\
& = & \I - \P_\X -\P_\X + \P_\X  \pause\\
& = & \I - \P_\X  \pause
    \end{eqnarray*}
\item Symmetry $\I - \P_\X = (\I - \P_\X)^T$  \pause
\item $\u \in C(\X)^{\perp} \Rightarrow \u \perp C(\X)$ and  $(\I
  -\P_\X) \u  = \u$  (projection)  \pause
\item if $\v \in C(\X)$, $(\I - \P_\X ) \v = \v - \v = 0$ 
  \end{itemize}
\end{frame}

\begin{frame} \frametitle{Null Space}
  \begin{itemize}
  \item   Column space $C(\X)$   \pause
  \item  $N(\A)$: Null space of $\A$ is  $\{ \u \mid \A \u = 0 \}$   \pause
  \item  Null space of $\X^T$ is $(\u \mid \X^T \u = 0 \}$  \pause
  \item  $ N(\X^T) = C(\X)^\perp$ the orthogonal complement of $C(\X)$  \pause 
  \item $N(\P_\X) = C(\X)^\perp$   \pause
  \end{itemize}
  \begin{eqnarray*}
\u \in N(\P) & \Rightarrow & \P_\X \u = \zero   \pause\\
& \Leftrightarrow &    \X(\X^T\X)^{-1} \X^T \u = \zero  \pause  \\
& \Leftrightarrow & \u^T\X(\X^T\X)^{-1} \X^T   = \zero^T   \pause \\
& \Leftrightarrow & \u^T\X(\X^T\X)^{-1} \X^T\b   = 0   \pause \\
& \Leftrightarrow & \u^T\v   = 0  \quad
   \v \in C(\X)\\ 
  \end{eqnarray*}
\end{frame}
%\section{MLEs}
\begin{frame} \frametitle{Models Again}
  \begin{itemize}
  \item   $\Y \sim \N(\mub, \sigma^2 \I_n)$ with $\mub \in C(\X)$  \pause
  \item   Claim: Maximum Likelihood Estimator (MLE) of $\mub$ is
    $\P_\X \Y$  \pause
\item Log Likelihood:  \pause

$$ \log \cL(\mub, \sigma^2) =
-\frac{n}{2} \log(\sigma^2) 
  - \frac 1 2 \frac{\| \Y - \mub \|^2}{\sigma^2} 
$$  \pause
\item Decompose $\Y = \P_\X \Y + (\I - \P_\X) \Y$  \pause
\item Use $\P_\X \mub = \mub$  \pause
\item and Simplify $$\| \Y - \mub \|^2$$
  \end{itemize}

\end{frame}
\end{document}

