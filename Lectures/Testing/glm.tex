\documentclass[]{beamer}
%\usepackage[dvips]{color}
\usepackage{graphicx}
\usepackage{amsmath,amssymb,array,comment,eucal}
\newcommand{\e}{\mathbf{e}}
\renewcommand{\P}{\mathbf{P}}
\newcommand{\F}{\mathbf{F}}
\newcommand{\R}{\textsf{R}}
\newcommand{\mat}[1] {\mathbf{#1}}
%\newcommand{\ind}{\mathrel{\mathop{\sim}\limits^{\mathit{ind}}}}
%\newcommand{\iid}{\mathrel{\mathop{\sim}\limits^{\mathit{iid}}}}
\newcommand{\E}{\textsf{E}}
\newcommand{\SE}{\textsf{SE}}
\newcommand{\SSE}{\textsf{SSE}}
\newcommand{\RSS}{\textsf{RSS}}
\newcommand{\FSS}{\textsf{FSS}}
\renewcommand{\SS}{\textsf{SS}}
\newcommand{\MSE}{\textsf{MSE}}
\newcommand{\SSR}{\textsf{SSR}}
\newcommand{\Be}{\textsf{Beta}}
\newcommand{\St}{\textsf{St}}
%\newcommand{\C}{\textsf{C}}
\newcommand{\GDP}{\textsf{GDP}}
\newcommand{\NcSt}{\textsf{NcSt}}
\newcommand{\Bin}{\textsf{Bin}}
\newcommand{\NB}{\textsf{NegBin}}
\renewcommand{\NG}{\textsf{NG}}
\newcommand{\N}{\textsf{N}}
\newcommand{\Ber}{\textsf{Ber}}
\newcommand{\Poi}{\text{Poi}}
\newcommand{\Gam}{\textsf{Gamma}}
\newcommand{\BB}{\textsf{BB}}
\newcommand{\Gm}{\textsf{G}}
\newcommand{\Un}{\textsf{Unif}}
\newcommand{\Ex}{\textsf{Exp}}
\newcommand{\DE}{\textsf{DE}}
\newcommand{\tr}{\textsf{tr}}
\newcommand{\cF}{{\cal{F}}}
\newcommand{\cL}{{\cal{L}}}
\newcommand{\cI}{{\cal{I}}}
\newcommand{\cB}{{\cal{B}}}
\newcommand{\cP}{{\cal{P}}}
\newcommand{\bbR}{\mathbb{R}}
\newcommand{\bbN}{\mathbb{N}}
\newcommand{\pperp}{\mathrel{{\rlap{$\,\perp$}\perp\,\,}}}
\newcommand{\OFP}{(\Omega,\cF, \P)}
\newcommand{\eps}{\boldsymbol{\epsilon}}
\newcommand{\1}{\mathbf{1}_n}
\newcommand{\gap}{\vspace{8mm}}
\newcommand{\ind}{\mathrel{\mathop{\sim}\limits^{\rm ind}}}
\newcommand{\simiid}{\ensuremath{\mathrel{\mathop{\sim}\limits^{\rm
iid}}}}
\newcommand{\eqindis}{\ensuremath{\mathrel{\mathop{=}\limits^{\rm D}}}}
\newcommand{\iid}{\textit{i.i.d.}}
\newcommand{\SSZ}{S_{zz}}
\newcommand{\SZW}{S_{zw}}
\newcommand{\Var}{\textsf{Var}}
\newcommand{\corr}{\textsf{corr}}
\newcommand{\diag}{\textsf{diag}}
\newcommand{\var}{\textsf{var}}
\newcommand{\Cov}{\textsf{Cov}}
\newcommand{\Sam}{{\cal S}}
\def\H{\mathbf{H}}
\newcommand{\I}{\mathbf{I}}
\newcommand{\Y}{\mathbf{Y}}
\newcommand{\tY}{\tilde{\mathbf{Y}}}
\newcommand{\Yhat}{\hat{\mathbf{Y}}}
\newcommand{\Yobs}{\mathbf{Y}_{{\cal S}}}
\newcommand{\barYobs}{\bar{Y}_{{\cal S}}}
\newcommand{\barYmiss}{\bar{Y}_{{\cal S}^c}}
\def\bv{\mathbf{b}}
\def\X{\mathbf{X}}
\def\tX{\tilde{\mathbf{X}}}
\def\x{\mathbf{x}}
\def\xbar{\bar{\mathbf{x}}}
\def\Xbar{\bar{\mathbf{X}}}
\def\Xg{\mathbf{X}_{\boldsymbol{\gamma}}}
\def\Ybar{\bar{\Y}}
\def\ybar{\bar{y}}
\def\y{\mathbf{y}}
\def\Yf{\mathbf{Y_f}}
\def\W{\mathbf{W}}
\def\L{\mathbf{L}}
\def\w{\mathbf{w}}
\def\U{\mathbf{U}}
\def\V{\mathbf{V}}
\def\Q{\mathbf{Q}}
\def\Z{\mathbf{Z}}
\def\z{\mathbf{z}}
\def\v{\mathbf{v}}
\def\u{\mathbf{u}}

\def\zero{\mathbf{0}}
\def\one{\mathbf{1}}
\newcommand{\taub}{\boldsymbol{\tau}}
\newcommand{\betav}{\boldsymbol{\beta}}
\newcommand{\alphav}{\boldsymbol{\alpha}}
\newcommand{\A}{\mathbf{A}}
\def\a{\mathbf{a}}
\def\K{\mathbf{K}}
\newcommand{\B}{\mathbf{B}}
\def\b{\boldsymbol{\beta}}
\def\bhat{\hat{\boldsymbol{\beta}}}
\def\btilde{\tilde{\boldsymbol{\beta}}}
\def\tb{\tilde{\boldsymbol{\beta}}}
\def\bg{\boldsymbol{\beta_\gamma}}
\def\bgnot{\boldsymbol{\beta_{(-\gamma)}}}
\def\mub{\boldsymbol{\mu}}
\def\tmub{\tilde{\boldsymbol{\mu}}}
\def\muhat{\hat{\boldsymbol{\mu}}}
\def\t{\boldsymbol{\theta}}
\def\tk{\boldsymbol{\theta}_k}
\def\tj{\boldsymbol{\theta}_j}
\def\Mk{\boldsymbol{{\cal M}}_k}
\def\M{\boldsymbol{{\cal M}}}
\def\Mj{\boldsymbol{{\cal M}}_j}
\def\Mi{\boldsymbol{{\cal M}}_i}
\def\Mg{{\boldsymbol{{\cal M}_\gamma}}}
\def\Mnull{\boldsymbol{{\cal M}}_{N}}
\def\gMPM{\boldsymbol{\gamma}_{\text{MPM}}}
\def\gHPM{\boldsymbol{\gamma}_{\text{HPM}}}
\def\Mfull{\boldsymbol{{\cal M}}_{F}}
\def\tg{\boldsymbol{\theta}_{\boldsymbol{\gamma}}}
\def\g{\boldsymbol{\gamma}}
\def\eg{\boldsymbol{\eta}_{\boldsymbol{\gamma}}}
\def\G{\mathbf{G}}
\def\cM{\cal M}
\def\D{\Delta}
\def \shat{{\hat{\sigma}}^2}
\def\uv{\mathbf{u}}
\def\l {\lambda}
\def\d{\delta}
\def\Sigmab{\boldsymbol{\Sigma}}
\def\Lambdab{\boldsymbol{\Lambda}}
\def\lambdab{\boldsymbol{\lambda}}
\def\Mg{{\cal M}_\gamma}
\def\S{{\cal{S}}}
\def\qg{p_{\boldsymbol{\gamma}}}
\def\pg{p_{\boldsymbol{\gamma}}}
\def\t{\boldsymbol{\theta}}  
\def\T{\boldsymbol{\Theta}}  
\usepackage{verbatim}

\usetheme{Warsaw}
\title{General Linear Model and Weighted Regression}
\subtitle{Merlise Clyde}
\author{STA721 Linear Models}
\institute{Duke University}
\date{September 25, 2012}
\logo{duke.eps}

\begin{document}
\maketitle

\begin{frame}\frametitle{Outline}
Topics
  \begin{itemize}
  \item Generalized Linear Model
  \item Maximum Likelihood Estimates
  \item Projections
  \item Weighted Regression
  \item Example

  \end{itemize}


Readings: Christensen Chapter 2 (section 7),  Chapter 10, Appendix B
\end{frame}
%\section{Models}
\begin{frame} \frametitle{General Linear Model}
 Generalized Linear Model:
 $$ \Y = \mub + \eps $$
  \begin{itemize}
  \item   $\mub \in C(\X) \Leftrightarrow \mub = \X \b$  \pause
  \item Assume $\E[\eps] = \zero_n$ and $\Cov[\eps] = \sigma^2 \V$,
    $\V > 0$ \pause  
 \item Generalized Least Squares estimate (GLS) $\bhat$ minimizes
$$ (\Y - \X \b)^T \V^{-1} (\Y - \X\b)$$  \pause
\item Add Normal Errors $\eps \sim \N(\zero, \sigma^2 \V)$ 
$$ \Y \sim \N(\X\b, \sigma^2 \V)$$
\item Then GLS is equivalent to Maximum
Likelihood Estimator  (MLE) of $\b$
\end{itemize}
\end{frame}


\begin{frame}
  \frametitle{Likelihood}
  \begin{itemize}
  \item Likelihood:  
$$ \cL(\mub, \sigma^2) \propto | \sigma^2 \V |^{-1/2} 
\exp \left\{  - \frac 1 2 \frac{(\Y - \mub)^T\V^{-1} (\Y - \mub) }{\sigma^2} \right\}
$$  \pause
\item $\V > 0$ write $\V = \Q \Q^T$  with $\Q > 0$ (Thm  B.22) \pause
\item $\V^{-1} = (\Q \Q^T)^{-1} = (\Q^T)^-1 \Q^{-1} = (\Q^{-1})^T
  \Q^{-1} = \Q^{-T} \Q^{-1}$ \pause
\item use notation $\Q^{-T} \equiv (\Q^{-1})^T = (\Q^T)^{-1}$ \pause
  \begin{eqnarray*}
\cL(\mub, \sigma^2) & \propto & | \sigma^2 \V |^{-1/2} 
e^{ \left\{  - \frac 1 2 \frac{(\Y - \mub)^T\Q^{-T} \Q^{-1}(\Y -
    \mub) }{\sigma^2} \right\}} \pause  \\ 
& \propto & | \sigma^2 \V |^{-1/2} 
e^{\left\{  - \frac 1 2 \frac{(\Q^{-1}\Y - \Q^{-1}\mub)^T (\Q^{-1}\Y -
    \Q^{-1}\mub) }{\sigma^2} \right\}}\pause \\
& \propto & | \sigma^2 \V |^{-1/2} 
e^{\left\{  - \frac 1 2 \frac{(\tY - \tmub)^T (\tY -
    \tmub) }{\sigma^2} \right\}} \\
  \end{eqnarray*}
where $\tY = \Q^{-1} \Y$ and  $\tmub = \Q^{-1} \mub$
 \end{itemize}
\end{frame}
\begin{frame}
  \frametitle{MLEs}
$$\cL(\tmub, \sigma^2) \propto  | \sigma^2 \V |^{-1/2} 
\exp{\left\{  - \frac 1 2 \frac{(\tY - \tmub)^T (\tY -
    \tmub) }{\sigma^2} \right\}}
$$  
\begin{itemize}
  \item $\tmub = \Q^{-1} \X \b$ so $\tmub \in C(\Q^{-1}\X) = C(\tX)$  \pause
  \item MLE of $\tmub$ is $\P_{\tX} \tY = \hat{\tmub}$ where $\P_{\tX} =
    \tX(\tX^T\tX)^{-1} \tX$ \pause
    \begin{eqnarray*}
\P_{\tX}\tY  & = & \Q^{-1} \X (\X^T\Q^{-T}\Q^{-1} \X)^{-1} \X^T
\Q^{-T} \Q^{-1} \Y    \pause  \\
  & = &  \Q^{-1} \X (\X^T \V^{-1} \X)^{-1} \X^T \V^{-1} \Y \pause  \\
  & = &\Q^{-1} \X (\X^T \V^{-1} \X)^{-1} \X^T \V^{-1} \Y \pause  \\
\Q \hat{\tmub} & = & \Q \Q^{-1}\X (\X^T \V^{-1} \X)^{-1} \X^T \V^{-1}
\Y \pause  \\
\Q \Q^{-1} \muhat & = &\X (\X^T \V^{-1} \X)^{-1} \X^T \V^{-1}
\Y  
    \end{eqnarray*}
\pause 
\item MLE $\muhat = \X (\X^T \V^{-1} \X)^{-1} \X^T \V^{-1}
\Y $ \pause 
\item MLE $\bhat = (\X^T \V^{-1} \X)^{-1} \X^T \V^{-1}\Y$ (full rank case)
  \end{itemize}
\end{frame}
\begin{frame}
  \frametitle{More on Projections}
  \begin{definition}
    \begin{enumerate}
    \item  $\P$ is a projection operator on $C(\P)$ along $N(\P)$ if $\P^2 =
 \P$ \pause
 \item Let $N$ and $M$ be two spaces such than $N \cap M = {\zero}$
   and $r(M) + r(N) = n$. Then the square matrix $\P$ is a projection
   on $M$ along $N$ if (i) $\P \v = \v$ for any $\v \in M$ and $\P \w
   = \zero$ for any $\w \in N$
    \end{enumerate} \pause
  \end{definition}


 \begin{itemize}
 \item  Null Space $N(\P) = \{\u:  \P \u = \zero \}$ \pause
  \item $\P_\X = \X (\X^T \X)^{-1} \X^T$ is a projection
  onto $C(\X)$ along  $N(\P)$  \pause
$$ \x \in N(\P) = \{ \x:  \P \x = 0 \} \Leftrightarrow \x^T \P^T = 0$$  \pause
$$\x \in C(\P^T)^\perp \pause = C(\P)^\perp \pause = C(\X)^\perp$$ \pause
\item   $\P_{\X} = \X (\X^T \V^{-1} \X)^{-1} \X^T \V^{-1}$ is also a projection
  onto $C(\X)$ \pause
\item $N(\P_\X): \{\u : \X(\X^T\V^{-1}\X)^{-1}\X^T \V^{-1} \u = \zero
  \} \Leftrightarrow \u^T \V^{-1} \X  = \zero$ \pause
\item Change inner product! 
 \end{itemize}
\end{frame}

\begin{frame}
  \frametitle{Inner Product}
  \begin{itemize}
  \item Usual Inner product $\langle \x, \y \rangle = \x^T\y$ \pause
 \item  Define inner product based on covariance structure
 $$\langle \x, \y \rangle_{\V^{-1}} = \x^T\V^{-1} \y$$ \pause
\item The adjoint of an operator $\A$ is defined as 
 $$\langle \x, \A\y \rangle_{\V^{-1}} = \langle \A^*\x, \y
 \rangle_{\V^{-1}}$$ \pause

  \end{itemize}
  \begin{definition}
 For an inner product space, $(\bbR^n, \langle \cdot, \cdot
 \rangle_{\V^{-1}})$ a projection operator $\P$ is an orthogonal
 projection operator onto $C(\P)$ along $C(\P)^\perp$ if \pause
 \begin{itemize}
 \item $\P = \P^2$  (projection) \pause
 \item $\P = \P^*$  (self-adjoint  ``symmetry'')  \pause
 \end{itemize}
  
  \end{definition}
Change definition of $\perp$ 
\end{frame}

\begin{frame}
  \frametitle{Show $\P_\X$ is Self-Adjoint}
 $\P_\X$ self-adjoint is
$$\langle \x, \P_\X \y \rangle_{\V^{-1}} = \langle \P_\X^*\x, \y
 \rangle_{\V^{-1}} =  \langle \P_\X\x, \y
 \rangle_{\V^{-1}}$$
\pause
 \begin{eqnarray*}
   \langle \x, \P_\X \y \rangle_{\V^{-1}} & = & \x^T \V^{-1} \P_\X
   \y  \pause\\ 
 & = & \x^T \V^{-1} \X (\X^T\V^{-1} \X)^{-1} \X^T \V^{-1}  \y  \pause\\ 
 & = &  (\P_\X \x)^T \V^{-1} \y  \pause\\
 & = &  \langle \P_\X\x, \y  \rangle_{\V^{-1}} \pause
 \end{eqnarray*}
$\u \in N(\P) \Leftrightarrow \u \in C(\X)^\perp$ where $\u^T \V^{-1}
\X =0$
\end{frame}
\begin{frame}
  \frametitle{MLEs}
  \begin{itemize}
  \item MLE of $\mub$: $\muhat = \P_\X \Y = \X (\X^T\V^{-1} \X)^{-1} \X^T \V^{-1}
    \Y$ \pause
\item MLE of $\b$ satisfies $\P_\X \Y = \X \bhat$ with 
$\bhat = (\X^T\V^{-1} \X)^{-1} \X^T \V^{-1} 
    \Y$ for full rank case \pause
\item $\Cov(\muhat) = \sigma^2 \P_\X \V \P_\X^T = \sigma^2 \X (\X^T
  \V^{-1} \X)^{-1} \X^T$ \pause
\item $\Cov(\bhat) = \sigma^2 (\X^T\V^{-1} \X)^{-1}$  (full rank case) \pause
\item $\muhat$ and $\bhat$ are BLUE - Best Linear Unbiased Estimators
  (Gauss-Markov Theorem with change in inner-product) \pause
\item With Normality, Minimum Variance Unbiased Estimators 
  \end{itemize}
\end{frame}
\begin{frame}
  \frametitle{Properties}
  \vspace{-24pt}
  \begin{itemize}
  \item $\P_\X^T \V^{-1} \P_\x = \P_\X^T \V^{-1} = \V^{-1} \P_\X$
    \pause

 \item $(\I- \P_\X)^T \V^{-1}(\I - \P_\x) = (\I - \P_\X)^T \V^{-1} =
   \V^{-1}(\I - \P_\X)$ \pause
\item  $(\I- \P_\X)^T \V^{-1}\P_\x = \zero$

  \end{itemize}
\vfill
\end{frame}


\begin{frame}
  \frametitle{Estimates of $\sigma^2$}
\vspace{-24pt}
Unbiased Estimate of $\sigma^2$ 
$$ \shat = \frac{\| (\I - \P_\X)\Y \|^2_{\V^{-1}}}{r(\I - \P_\X)} =
\frac{\Y^T(\I - \P_{\X})^T \V^{-1}   (\I - \P_{\X}) \Y}{n - r(\X)}$$ 
$$\E[\Y^T(\I - \P_{\X})^T \V^{-1}   (\I - \P_{\X}) \Y] = n - r(\X)$$
\vfill
\end{frame}

\begin{frame}
  \frametitle{Distributions}
  \begin{theorem}
Let $\Y \sim \N(\mub, \sigma^2\V)$.
If $\P$ is a rank $k$ orthogonal projection on $C(\P)$ with
inner product $\langle \cdot, \cdot \rangle_{\V^-1}$ then 
$$ \frac{\| \P \Y \|^2_{\V^{-1}}}{\sigma^2} \sim \chi^2_{k}
\left( \| \P \mub \|^2_{\V^{-1}}\right)$$  
  \end{theorem} \pause


  \begin{itemize}
  \item Model Comparison:  H$_o: \mub \in C(\X_0)$ vs H$_a: \mub \in
    C(\X)$ \pause
  \item Extra Sum of Squares:  
$$ \|\Y\|^2_{\v^{-1}} = \|\P_{\X_0}\Y\|^2_{\v^{-1}} + \|( \P_\X -
\P_{\X_0})\Y\|^2_{\v^{-1}} + \|(\I - \P_{\X})\Y\|^2_{\v^{-1}}$$ \pause
\item  F test of $H_0$  (distribution under null)
$$  \frac{\|( \P_\X - \P_{\X_0}) \Y \|^2_{\V^{-1}}/(r(P_\X) - r(P{\X_0}))}{\| (\I - \P_\X)\Y
  \|^2_{\V^{-1}}/(n - r(\X)) } \sim F(r(X), n-r(X)) $$
    
\end{itemize}


\end{frame}
\begin{frame}
  \frametitle{Weighted Regression}
Special Case: Take $\V$ diagonal with elements $w_i^{-1}$ $\V^{-1} = \W$

Weighted Least Squares is Generalized Least Squares 
$$\bhat = (\X^T \W \X)^{-1} \X^T\W\Y$$

Inner product $\u^T \W \v = \sum_i w_i u_i v_i$
\end{frame}

\begin{frame}
  \frametitle{Climate Change ?}
  Scientists are interested in the Earth's temperature change since the last 
glacial maximum, about 20,000 years ago. \pause
\begin{itemize}
\item  The first study to estimate the 
temperature change was published in 1980 \pause
\item  Estimated a change of -1.5 
degrees C, $\pm 1.2$ degrees C in tropical sea surface temperatures.  \pause
\item The 
negative value means that the Earth was colder then than now.  \pause
\item Since 1980 
there have been many other studies, which use different 
measurement techniques, or proxies.  \pause
\item Some proxies can be used over land, 
others over water. 
\end{itemize}


\end{frame}

\begin{frame}
  \frametitle{Proxies}
   The proxies are

   \begin{enumerate}
   \item      "Mg/Ca"           1
   \item     "alkenone"        2
   \item        "Faunal"          3
   \item             "Sr/Ca"           4
   \item            "del 180"         5
   \item             "Ice Core"        6
   \item             "Pollen"          7
   \item             "Noble Gas"       8
   \end{enumerate}

\end{frame}
\begin{frame}[fragile]
  \frametitle{Data}
  \begin{tiny}
\begin{verbatim}
  climate =
  read.table("http://www.stat.duke.edu/courses/Fall10/sta290/datasets/climate.dat",

  header=T)  
\end{verbatim}
    
  \end{tiny}
\pause
Each study reports
\begin{itemize}
\item {\tt deltaT} an estimate of the temperature change \pause
\item {\tt sdev}  a standard deviation of that
estimate \pause
\item {\tt proxy}  the proxy used (coded 1 to 8),  \pause
\item {\tt T.M} whether it was a
terrestrial or marine study (T/M), which is coded as 0 for
Terrestrial, 1 for Marine,  \pause
\item {\tt latitude} at which data were collected 
\end{itemize}
\end{frame}

\begin{frame}[fragile]
  \frametitle{Anova}
  \begin{small}
\begin{verbatim}
climate.lm = lm(deltaT ~ proxy *(poly(latitude,2)),
                weights=(1/sdev^2), 
                data=climate)
anova(climate.lm)
Response: deltaT
                        Df  Sum Sq Mean Sq F value    Pr(>F)    
proxy                    7 307.598  43.943  9.8541 3.848e-07 ***
poly(latitude, 2)        2  10.457   5.228  1.1725    0.3198    
proxy:poly(latitude, 2) 12  74.065   6.172  1.3841    0.2126    
Residuals               41 182.833   4.459  
\end{verbatim}
    
  \end{small}
Sequential Sum of Squares 
\end{frame}

\begin{frame}[fragile]
 \frametitle{Sequential Sum of Squares}
   \begin{small}
\begin{verbatim}
>anova(lm(deltaT ~ (poly(latitude,2))* proxy, weights=1/sdev^2,
          data=climate))
 Analysis of Variance Table

Response: deltaT
                        Df  Sum Sq Mean Sq F value    Pr(>F)    
poly(latitude, 2)        2  79.869  39.935  8.9553 0.0005931 ***
proxy                    7 238.185  34.026  7.6304  6.93e-06 ***
poly(latitude, 2):proxy 12  74.065   6.172  1.3841 0.2125512    
Residuals               41 182.833   4.459                      
                   
---
Signif. codes:  0 '***' 0.001 '**' 0.01 '*' 0.05 '.' 0.1 ' ' 1 
 
\end{verbatim}
\end{small}
\end{frame}

\end{document}

