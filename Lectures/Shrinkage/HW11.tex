\documentclass[12pt]{article}
\usepackage{fullpage, amssymb,amsmath}
\pagestyle{empty}
\newcommand{\e}{\mathbf{e}}
\renewcommand{\P}{\mathbf{P}}
\newcommand{\F}{\mathbf{F}}
\newcommand{\R}{\textsf{R}}
\newcommand{\mat}[1] {\mathbf{#1}}
%\newcommand{\ind}{\mathrel{\mathop{\sim}\limits^{\mathit{ind}}}}
%\newcommand{\iid}{\mathrel{\mathop{\sim}\limits^{\mathit{iid}}}}
\newcommand{\E}{\textsf{E}}
\newcommand{\SE}{\textsf{SE}}
\newcommand{\SSE}{\textsf{SSE}}
\newcommand{\RSS}{\textsf{RSS}}
\newcommand{\FSS}{\textsf{FSS}}
\renewcommand{\SS}{\textsf{SS}}
\newcommand{\MSE}{\textsf{MSE}}
\newcommand{\SSR}{\textsf{SSR}}
\newcommand{\Be}{\textsf{Beta}}
\newcommand{\St}{\textsf{St}}
%\newcommand{\C}{\textsf{C}}
\newcommand{\GDP}{\textsf{GDP}}
\newcommand{\NcSt}{\textsf{NcSt}}
\newcommand{\Bin}{\textsf{Bin}}
\newcommand{\NB}{\textsf{NegBin}}
\renewcommand{\NG}{\textsf{NG}}
\newcommand{\N}{\textsf{N}}
\newcommand{\Ber}{\textsf{Ber}}
\newcommand{\Poi}{\text{Poi}}
\newcommand{\Gam}{\textsf{Gamma}}
\newcommand{\BB}{\textsf{BB}}
\newcommand{\Gm}{\textsf{G}}
\newcommand{\Un}{\textsf{Unif}}
\newcommand{\Ex}{\textsf{Exp}}
\newcommand{\DE}{\textsf{DE}}
\newcommand{\tr}{\textsf{tr}}
\newcommand{\cF}{{\cal{F}}}
\newcommand{\cL}{{\cal{L}}}
\newcommand{\cI}{{\cal{I}}}
\newcommand{\cB}{{\cal{B}}}
\newcommand{\cP}{{\cal{P}}}
\newcommand{\bbR}{\mathbb{R}}
\newcommand{\bbN}{\mathbb{N}}
\newcommand{\pperp}{\mathrel{{\rlap{$\,\perp$}\perp\,\,}}}
\newcommand{\OFP}{(\Omega,\cF, \P)}
\newcommand{\eps}{\boldsymbol{\epsilon}}
\newcommand{\1}{\mathbf{1}_n}
\newcommand{\gap}{\vspace{8mm}}
\newcommand{\ind}{\mathrel{\mathop{\sim}\limits^{\rm ind}}}
\newcommand{\simiid}{\ensuremath{\mathrel{\mathop{\sim}\limits^{\rm
iid}}}}
\newcommand{\eqindis}{\ensuremath{\mathrel{\mathop{=}\limits^{\rm D}}}}
\newcommand{\iid}{\textit{i.i.d.}}
\newcommand{\SSZ}{S_{zz}}
\newcommand{\SZW}{S_{zw}}
\newcommand{\Var}{\textsf{Var}}
\newcommand{\corr}{\textsf{corr}}
\newcommand{\diag}{\textsf{diag}}
\newcommand{\var}{\textsf{var}}
\newcommand{\Cov}{\textsf{Cov}}
\newcommand{\Sam}{{\cal S}}
\def\H{\mathbf{H}}
\newcommand{\I}{\mathbf{I}}
\newcommand{\Y}{\mathbf{Y}}
\newcommand{\tY}{\tilde{\mathbf{Y}}}
\newcommand{\Yhat}{\hat{\mathbf{Y}}}
\newcommand{\Yobs}{\mathbf{Y}_{{\cal S}}}
\newcommand{\barYobs}{\bar{Y}_{{\cal S}}}
\newcommand{\barYmiss}{\bar{Y}_{{\cal S}^c}}
\def\bv{\mathbf{b}}
\def\X{\mathbf{X}}
\def\tX{\tilde{\mathbf{X}}}
\def\x{\mathbf{x}}
\def\xbar{\bar{\mathbf{x}}}
\def\Xbar{\bar{\mathbf{X}}}
\def\Xg{\mathbf{X}_{\boldsymbol{\gamma}}}
\def\Ybar{\bar{\Y}}
\def\ybar{\bar{y}}
\def\y{\mathbf{y}}
\def\Yf{\mathbf{Y_f}}
\def\W{\mathbf{W}}
\def\L{\mathbf{L}}
\def\w{\mathbf{w}}
\def\U{\mathbf{U}}
\def\V{\mathbf{V}}
\def\Q{\mathbf{Q}}
\def\Z{\mathbf{Z}}
\def\z{\mathbf{z}}
\def\v{\mathbf{v}}
\def\u{\mathbf{u}}

\def\zero{\mathbf{0}}
\def\one{\mathbf{1}}
\newcommand{\taub}{\boldsymbol{\tau}}
\newcommand{\betav}{\boldsymbol{\beta}}
\newcommand{\alphav}{\boldsymbol{\alpha}}
\newcommand{\A}{\mathbf{A}}
\def\a{\mathbf{a}}
\def\K{\mathbf{K}}
\newcommand{\B}{\mathbf{B}}
\def\b{\boldsymbol{\beta}}
\def\bhat{\hat{\boldsymbol{\beta}}}
\def\btilde{\tilde{\boldsymbol{\beta}}}
\def\tb{\tilde{\boldsymbol{\beta}}}
\def\bg{\boldsymbol{\beta_\gamma}}
\def\bgnot{\boldsymbol{\beta_{(-\gamma)}}}
\def\mub{\boldsymbol{\mu}}
\def\tmub{\tilde{\boldsymbol{\mu}}}
\def\muhat{\hat{\boldsymbol{\mu}}}
\def\t{\boldsymbol{\theta}}
\def\tk{\boldsymbol{\theta}_k}
\def\tj{\boldsymbol{\theta}_j}
\def\Mk{\boldsymbol{{\cal M}}_k}
\def\M{\boldsymbol{{\cal M}}}
\def\Mj{\boldsymbol{{\cal M}}_j}
\def\Mi{\boldsymbol{{\cal M}}_i}
\def\Mg{{\boldsymbol{{\cal M}_\gamma}}}
\def\Mnull{\boldsymbol{{\cal M}}_{N}}
\def\gMPM{\boldsymbol{\gamma}_{\text{MPM}}}
\def\gHPM{\boldsymbol{\gamma}_{\text{HPM}}}
\def\Mfull{\boldsymbol{{\cal M}}_{F}}
\def\tg{\boldsymbol{\theta}_{\boldsymbol{\gamma}}}
\def\g{\boldsymbol{\gamma}}
\def\eg{\boldsymbol{\eta}_{\boldsymbol{\gamma}}}
\def\G{\mathbf{G}}
\def\cM{\cal M}
\def\D{\Delta}
\def \shat{{\hat{\sigma}}^2}
\def\uv{\mathbf{u}}
\def\l {\lambda}
\def\d{\delta}
\def\Sigmab{\boldsymbol{\Sigma}}
\def\Lambdab{\boldsymbol{\Lambda}}
\def\lambdab{\boldsymbol{\lambda}}
\def\Mg{{\cal M}_\gamma}
\def\S{{\cal{S}}}
\def\qg{p_{\boldsymbol{\gamma}}}
\def\pg{p_{\boldsymbol{\gamma}}}
\def\t{\boldsymbol{\theta}}  
\def\T{\boldsymbol{\Theta}}  
\begin{document}
{\bf STA721  }
\vspace{.1in}
\begin{center}
{\large \bf Homework 11} \\
\end{center}
\vspace{.5in}
\noindent

\begin{enumerate}
\item Derive the full conditionals in Casella and Park (2008) see
  website for link to paper.
\item  As a variation on the simulation study in Nott \& Kohn (Biometrika
2005) (nott-kohn.R), we will explore shrinkage estimators in the normal linear model 
\begin{equation}
  \label{eq:model}
\Y \sim \N(\X \betav, \I_n\sigma^2)
\end{equation}
where $\X$ has been generated to have a given correlation structure
(see the R code nott-kohn.R in the Shrinkage section).  Two of the
variables have a correlation of near 0.99, with the others more
modest.  Of the 20 variables, only 8 are related to $\Y$.


\begin{enumerate}
\item Calculate the $E[(\hat{\beta} - \beta)^T(\hat{\beta} - \beta)]$,
  the expected MSE for OLS under the full model. 
\item For each simulation, the OLS coefficients are found and an
  observed MSE = $(\hat{\beta}^{(s)} - \beta)^T(\hat{\beta}^{(s)} -
  \beta)$ is computed for each of the $s$ simulated datasets.  Does
  the average of the vector of observed MSEs provide a good estimate
  of the average of $E[(\hat{\beta} - \beta)^T(\hat{\beta} - \beta)]$?
  What does the distribution of MSEs look like?  Do you think you need
  to use a larger number of simulations?    

\item Modify the R-code to use lasso (lars), ridge regression
  (lm.ridge from MASS or other), and the horseshoe (bhs from monomvn package on CRAN) to
  estimate $\beta$ (be careful about which methods standardize
  variables)).  In terms of MSE, which method appears to be best (look
  at average MSE and side-by-side boxplots)?  Which method has the
  least bias? (most variance?)  How do they compare to OLS?  Because
  the methods are compard on the same simulated data, we can use
  ``blocking'' to eliminate some of the MC variation. For each
  simulated data set, take the MSE for all the methods and divide by
  the smallest MSE for that simulation (hint: use {\tt apply} and {\tt
    sweep}) and then look at side-by-side boxplots of the relative MSE
  -- those closest to 1 are best.
\item (Optional)- repeat the above, but consider predictive MSE for
  predicting new $\Y^*$'s at new $\X^*$ values with the same correlation
  structure.  Are the methods that are best for estimating $\b$ also
  best for estimating $\Y^*$
\end{enumerate}
\end{enumerate}
\end{document}

Assume  the model
\begin{equation}
  \label{eq:true}c
\Y = 1 \alpha + \X \betav + \eps 
\end{equation}
where $\X$ is $n \times p$ and
full column rank for the problems below.

\begin{enumerate}


\item For $\muhat = \X \betahat$ and $\mub = \X \betav$, find the
  expected loss
  $\E[\| \muhat - \mub \|^2$
  where the expectation is taken with respect to the distribution of
  the data $\Y$ given in (1). (Hint: re-express so that you may use
  the results about the expectation of a quadratic
  form. 

\item  Define $\mupost = \X \betapost$ where $\betapost$ is the
  posterior mean under the Zellner $g$-prior.  Find the sampling
  distribution of $\mupost$.  (As a function of $\Y$, what is the
  distribution of $\mupost$ under the model in (1)).
  
   Is the posterior mean $\mupost$ 
   unbiased for estimating $\mub$?  If not, what is the bias?

\item Find $\E[\| \mupost - \mub \|^2$ assuming model (1) and express
  as a function of $p$, $g$ and $\|\mu\|^2$.  This expectation should
  be taken with respect to the sampling distribution of $\Y$ not the
  posterior distribution of $\mub$.

%\item The Gauss-Markov Theorem showed that
%  out of the class of unbiased linear estimators, the MLE has the
%  smallest variance.  If we use the posterior means above, can they have a
%  smaller loss than  the MLE for estimating $\mub$?  Can it be much worse? Expl%ain.


 \end{enumerate}

\end{enumerate}
\end{document}