\documentclass{article}


\usepackage{graphicx, fullpage}
\usepackage{amsmath,amssymb,array,comment,eucal}
\newcommand{\e}{\mathbf{e}}
\renewcommand{\P}{\mathbf{P}}
\newcommand{\F}{\mathbf{F}}
\newcommand{\R}{\textsf{R}}
\newcommand{\mat}[1] {\mathbf{#1}}
%\newcommand{\ind}{\mathrel{\mathop{\sim}\limits^{\mathit{ind}}}}
%\newcommand{\iid}{\mathrel{\mathop{\sim}\limits^{\mathit{iid}}}}
\newcommand{\E}{\textsf{E}}
\newcommand{\SE}{\textsf{SE}}
\newcommand{\SSE}{\textsf{SSE}}
\newcommand{\RSS}{\textsf{RSS}}
\newcommand{\FSS}{\textsf{FSS}}
\renewcommand{\SS}{\textsf{SS}}
\newcommand{\MSE}{\textsf{MSE}}
\newcommand{\SSR}{\textsf{SSR}}
\newcommand{\Be}{\textsf{Beta}}
\newcommand{\St}{\textsf{St}}
%\newcommand{\C}{\textsf{C}}
\newcommand{\GDP}{\textsf{GDP}}
\newcommand{\NcSt}{\textsf{NcSt}}
\newcommand{\Bin}{\textsf{Bin}}
\newcommand{\NB}{\textsf{NegBin}}
\renewcommand{\NG}{\textsf{NG}}
\newcommand{\N}{\textsf{N}}
\newcommand{\Ber}{\textsf{Ber}}
\newcommand{\Poi}{\text{Poi}}
\newcommand{\Gam}{\textsf{Gamma}}
\newcommand{\BB}{\textsf{BB}}
\newcommand{\Gm}{\textsf{G}}
\newcommand{\Un}{\textsf{Unif}}
\newcommand{\Ex}{\textsf{Exp}}
\newcommand{\DE}{\textsf{DE}}
\newcommand{\tr}{\textsf{tr}}
\newcommand{\cF}{{\cal{F}}}
\newcommand{\cL}{{\cal{L}}}
\newcommand{\cI}{{\cal{I}}}
\newcommand{\cB}{{\cal{B}}}
\newcommand{\cP}{{\cal{P}}}
\newcommand{\bbR}{\mathbb{R}}
\newcommand{\bbN}{\mathbb{N}}
\newcommand{\pperp}{\mathrel{{\rlap{$\,\perp$}\perp\,\,}}}
\newcommand{\OFP}{(\Omega,\cF, \P)}
\newcommand{\eps}{\boldsymbol{\epsilon}}
\newcommand{\1}{\mathbf{1}_n}
\newcommand{\gap}{\vspace{8mm}}
\newcommand{\ind}{\mathrel{\mathop{\sim}\limits^{\rm ind}}}
\newcommand{\simiid}{\ensuremath{\mathrel{\mathop{\sim}\limits^{\rm
iid}}}}
\newcommand{\eqindis}{\ensuremath{\mathrel{\mathop{=}\limits^{\rm D}}}}
\newcommand{\iid}{\textit{i.i.d.}}
\newcommand{\SSZ}{S_{zz}}
\newcommand{\SZW}{S_{zw}}
\newcommand{\Var}{\textsf{Var}}
\newcommand{\corr}{\textsf{corr}}
\newcommand{\diag}{\textsf{diag}}
\newcommand{\var}{\textsf{var}}
\newcommand{\Cov}{\textsf{Cov}}
\newcommand{\Sam}{{\cal S}}
\def\H{\mathbf{H}}
\newcommand{\I}{\mathbf{I}}
\newcommand{\Y}{\mathbf{Y}}
\newcommand{\tY}{\tilde{\mathbf{Y}}}
\newcommand{\Yhat}{\hat{\mathbf{Y}}}
\newcommand{\Yobs}{\mathbf{Y}_{{\cal S}}}
\newcommand{\barYobs}{\bar{Y}_{{\cal S}}}
\newcommand{\barYmiss}{\bar{Y}_{{\cal S}^c}}
\def\bv{\mathbf{b}}
\def\X{\mathbf{X}}
\def\tX{\tilde{\mathbf{X}}}
\def\x{\mathbf{x}}
\def\xbar{\bar{\mathbf{x}}}
\def\Xbar{\bar{\mathbf{X}}}
\def\Xg{\mathbf{X}_{\boldsymbol{\gamma}}}
\def\Ybar{\bar{\Y}}
\def\ybar{\bar{y}}
\def\y{\mathbf{y}}
\def\Yf{\mathbf{Y_f}}
\def\W{\mathbf{W}}
\def\L{\mathbf{L}}
\def\w{\mathbf{w}}
\def\U{\mathbf{U}}
\def\V{\mathbf{V}}
\def\Q{\mathbf{Q}}
\def\Z{\mathbf{Z}}
\def\z{\mathbf{z}}
\def\v{\mathbf{v}}
\def\u{\mathbf{u}}

\def\zero{\mathbf{0}}
\def\one{\mathbf{1}}
\newcommand{\taub}{\boldsymbol{\tau}}
\newcommand{\betav}{\boldsymbol{\beta}}
\newcommand{\alphav}{\boldsymbol{\alpha}}
\newcommand{\A}{\mathbf{A}}
\def\a{\mathbf{a}}
\def\K{\mathbf{K}}
\newcommand{\B}{\mathbf{B}}
\def\b{\boldsymbol{\beta}}
\def\bhat{\hat{\boldsymbol{\beta}}}
\def\btilde{\tilde{\boldsymbol{\beta}}}
\def\tb{\tilde{\boldsymbol{\beta}}}
\def\bg{\boldsymbol{\beta_\gamma}}
\def\bgnot{\boldsymbol{\beta_{(-\gamma)}}}
\def\mub{\boldsymbol{\mu}}
\def\tmub{\tilde{\boldsymbol{\mu}}}
\def\muhat{\hat{\boldsymbol{\mu}}}
\def\t{\boldsymbol{\theta}}
\def\tk{\boldsymbol{\theta}_k}
\def\tj{\boldsymbol{\theta}_j}
\def\Mk{\boldsymbol{{\cal M}}_k}
\def\M{\boldsymbol{{\cal M}}}
\def\Mj{\boldsymbol{{\cal M}}_j}
\def\Mi{\boldsymbol{{\cal M}}_i}
\def\Mg{{\boldsymbol{{\cal M}_\gamma}}}
\def\Mnull{\boldsymbol{{\cal M}}_{N}}
\def\gMPM{\boldsymbol{\gamma}_{\text{MPM}}}
\def\gHPM{\boldsymbol{\gamma}_{\text{HPM}}}
\def\Mfull{\boldsymbol{{\cal M}}_{F}}
\def\tg{\boldsymbol{\theta}_{\boldsymbol{\gamma}}}
\def\g{\boldsymbol{\gamma}}
\def\eg{\boldsymbol{\eta}_{\boldsymbol{\gamma}}}
\def\G{\mathbf{G}}
\def\cM{\cal M}
\def\D{\Delta}
\def \shat{{\hat{\sigma}}^2}
\def\uv{\mathbf{u}}
\def\l {\lambda}
\def\d{\delta}
\def\Sigmab{\boldsymbol{\Sigma}}
\def\Lambdab{\boldsymbol{\Lambda}}
\def\lambdab{\boldsymbol{\lambda}}
\def\Mg{{\cal M}_\gamma}
\def\S{{\cal{S}}}
\def\qg{p_{\boldsymbol{\gamma}}}
\def\pg{p_{\boldsymbol{\gamma}}}
\def\t{\boldsymbol{\theta}}  
\def\T{\boldsymbol{\Theta}}  
\usepackage{verbatim}

\begin{document}
\begin{center}
  STA 721 HW 10
\end{center}

\begin{enumerate}
\item Verify  equation (3) starting from  (1) for the
  rotated model for $\Y^*$: 
  \begin{align}
    \Y & = \one \alpha + \X^s \b + \eps \\
    \U^T\Y & = \U^T\one \alpha + \U^T \X^s \b + \U^T\eps \\
\left(
    \begin{array}{c}
      y_0^* \\
      y_1^* \\
    \vdots \\
      y_p^* \\
      y_{p+1}^* \\
      \vdots \\
      y_{n-1}^*
    \end{array} \right)
 & =  
\left[
 \begin{array}{ccccc}
      \sqrt{n} & 0 & 0  & \ldots & 0\\
      0 & l_1 & 0 & \ldots & 0 \\
    \vdots & 0 & \ddots &  0 & \ldots \\
    \vdots&  0 & \vdots & \ddots & 0 \\
    \vdots&  \vdots & \vdots & 0 & l_p \\
    \vdots & \vdots & \vdots &\vdots &  0 \\
    \vdots & \vdots & \vdots &\vdots &  \vdots \\
     0 & \ldots & \ldots & \ldots & 0  
    \end{array} \right]
\left(
                                    \begin{array}{c}
                                      \alpha \\
                                      \gamma_1 \\
                                      \vdots \\
                                      \gamma_p
                                    \end{array}
\right)
+ \eps^*
  \end{align} 
where$\X^s$ has been centered and standardized so that each column has
length 1 and mean 0 (see scale in R), $\U = [\one_n/\sqrt{n} \, \U_p \,
\U_{n-p-1}]$ is an orthogonal matrix with $\X^s = \U_p L \V^T$ from
the singular value decomposition, and $\g = \V^T\b$.

\item Show that OLS estimates $\hat{\alpha} = \ybar$ and
  $\hat{\gamma}_i = y_i^*/l_i$ for $i = 1, \ldots, p$.

\item Show that $\SSE = \Y^T\U_{n-p-1}\U_{n-p-1}^T \Y = \Y(\I -
  \frac{1}{n}\one \one^T - \U_p \U_p^T)\Y$.  Do you need to find
  $\U_{n-p-1}^T\Y$ to compute SSE?  Can you express SSE as a function of
  the MLEs of  $\alpha$ and $\hat{\gamma}$s (and $\Y^TY$)?
\item Write the  likelihood function as a product of three terms:
a part that involves only $\alpha \mid \phi$, $\gamma \mid \phi$ and
$\phi$ with the MLES and SSE using $\Y^*$. 

\item Derive  the
  full conditional distributions  for  $\alpha$,   $\gamma$, $\kappa_j$
  and $\phi$ assuming 
  \begin{align}
    p(\alpha, \phi) & \propto 1/\phi \\
    \gamma_j \mid \kappa_j, \phi, \alpha & \simiid \N(0, \frac{1}{\phi
                                           \kappa_j}) \\
   \kappa_j & \simiid G(1/2, 1/2)
  \end{align}
You should have a name for the distribution and  
  expressions for all hyperparameters, not just an expression for the
  density. Hint: write down likelihoods and priors,
  but ignore any terms that do not involve the parameter of interest
  (they go into constant of proportionality).  Simplify until you
  recognize the distribution. 

\item Modify your Gamma prior on $\kappa_i$ to try to capture the
  desired features of Goldstein \& Smith (1974)  (See Christensen
  Chapter 15) where if
$$
\gamma_i^2 < \sigma^2\left[ \frac{2}{\kappa_i} + \frac{1}{l_i^2}  \right]
$$ the fixed $\kappa_i$ Generalized Ridge shrinkage estimator (posterior mean) beats OLS:
\begin{itemize}
\item 
If $l_i$ is small almost any $\kappa_i$ will improve over OLS 
\item 
if $l_i^2$ is large then only very small values of $\kappa_i$ will give an
improvement
based on $l_i$, i.e. if $l_i^2$ is large $kappa_i$ should be small.
Plot your prior densities, with an overlay of the G(1/2, 1/2).
\end{itemize}
\item Find the updated full conditionals based on your choice above.
  Do you need to update all of the full conditionals?  Explain.

\item Implement your 2 models in R, JAGS or other language (see
  earlier JAGS code as a starting point) and apply this to the {\tt
    longley} data.  How do your results compare to classical ridge?
  Include histograms of the posterior distributions of coefficients,
  plus means and credible intervals, as well as histograms of the
  $\kappa$'s with the prior density overlaid.  How sensitive are the
  results to the prior assumptions?  How do the estimates of
  $\kappa_i$ compare to the best GCV estimate from class?

\item Explain the computational advantage of using the canonical
  parameterization in MCMC.
\end{enumerate}
\end{document}
