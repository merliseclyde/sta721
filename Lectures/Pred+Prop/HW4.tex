\documentclass{article}

\usepackage{fullpage}
\usepackage{amsmath,amssymb,array,comment,eucal}
\pagestyle{empty}
\newcommand{\e}{\mathbf{e}}
\renewcommand{\P}{\mathbf{P}}
\newcommand{\F}{\mathbf{F}}
\newcommand{\R}{\textsf{R}}
\newcommand{\mat}[1] {\mathbf{#1}}
%\newcommand{\ind}{\mathrel{\mathop{\sim}\limits^{\mathit{ind}}}}
%\newcommand{\iid}{\mathrel{\mathop{\sim}\limits^{\mathit{iid}}}}
\newcommand{\E}{\textsf{E}}
\newcommand{\SE}{\textsf{SE}}
\newcommand{\SSE}{\textsf{SSE}}
\newcommand{\RSS}{\textsf{RSS}}
\newcommand{\FSS}{\textsf{FSS}}
\renewcommand{\SS}{\textsf{SS}}
\newcommand{\MSE}{\textsf{MSE}}
\newcommand{\SSR}{\textsf{SSR}}
\newcommand{\Be}{\textsf{Beta}}
\newcommand{\St}{\textsf{St}}
%\newcommand{\C}{\textsf{C}}
\newcommand{\GDP}{\textsf{GDP}}
\newcommand{\NcSt}{\textsf{NcSt}}
\newcommand{\Bin}{\textsf{Bin}}
\newcommand{\NB}{\textsf{NegBin}}
\renewcommand{\NG}{\textsf{NG}}
\newcommand{\N}{\textsf{N}}
\newcommand{\Ber}{\textsf{Ber}}
\newcommand{\Poi}{\text{Poi}}
\newcommand{\Gam}{\textsf{Gamma}}
\newcommand{\BB}{\textsf{BB}}
\newcommand{\Gm}{\textsf{G}}
\newcommand{\Un}{\textsf{Unif}}
\newcommand{\Ex}{\textsf{Exp}}
\newcommand{\DE}{\textsf{DE}}
\newcommand{\tr}{\textsf{tr}}
\newcommand{\cF}{{\cal{F}}}
\newcommand{\cL}{{\cal{L}}}
\newcommand{\cI}{{\cal{I}}}
\newcommand{\cB}{{\cal{B}}}
\newcommand{\cP}{{\cal{P}}}
\newcommand{\bbR}{\mathbb{R}}
\newcommand{\bbN}{\mathbb{N}}
\newcommand{\pperp}{\mathrel{{\rlap{$\,\perp$}\perp\,\,}}}
\newcommand{\OFP}{(\Omega,\cF, \P)}
\newcommand{\eps}{\boldsymbol{\epsilon}}
\newcommand{\1}{\mathbf{1}_n}
\newcommand{\gap}{\vspace{8mm}}
\newcommand{\ind}{\mathrel{\mathop{\sim}\limits^{\rm ind}}}
\newcommand{\simiid}{\ensuremath{\mathrel{\mathop{\sim}\limits^{\rm
iid}}}}
\newcommand{\eqindis}{\ensuremath{\mathrel{\mathop{=}\limits^{\rm D}}}}
\newcommand{\iid}{\textit{i.i.d.}}
\newcommand{\SSZ}{S_{zz}}
\newcommand{\SZW}{S_{zw}}
\newcommand{\Var}{\textsf{Var}}
\newcommand{\corr}{\textsf{corr}}
\newcommand{\diag}{\textsf{diag}}
\newcommand{\var}{\textsf{var}}
\newcommand{\Cov}{\textsf{Cov}}
\newcommand{\Sam}{{\cal S}}
\def\H{\mathbf{H}}
\newcommand{\I}{\mathbf{I}}
\newcommand{\Y}{\mathbf{Y}}
\newcommand{\tY}{\tilde{\mathbf{Y}}}
\newcommand{\Yhat}{\hat{\mathbf{Y}}}
\newcommand{\Yobs}{\mathbf{Y}_{{\cal S}}}
\newcommand{\barYobs}{\bar{Y}_{{\cal S}}}
\newcommand{\barYmiss}{\bar{Y}_{{\cal S}^c}}
\def\bv{\mathbf{b}}
\def\X{\mathbf{X}}
\def\tX{\tilde{\mathbf{X}}}
\def\x{\mathbf{x}}
\def\xbar{\bar{\mathbf{x}}}
\def\Xbar{\bar{\mathbf{X}}}
\def\Xg{\mathbf{X}_{\boldsymbol{\gamma}}}
\def\Ybar{\bar{\Y}}
\def\ybar{\bar{y}}
\def\y{\mathbf{y}}
\def\Yf{\mathbf{Y_f}}
\def\W{\mathbf{W}}
\def\L{\mathbf{L}}
\def\w{\mathbf{w}}
\def\U{\mathbf{U}}
\def\V{\mathbf{V}}
\def\Q{\mathbf{Q}}
\def\Z{\mathbf{Z}}
\def\z{\mathbf{z}}
\def\v{\mathbf{v}}
\def\u{\mathbf{u}}

\def\zero{\mathbf{0}}
\def\one{\mathbf{1}}
\newcommand{\taub}{\boldsymbol{\tau}}
\newcommand{\betav}{\boldsymbol{\beta}}
\newcommand{\alphav}{\boldsymbol{\alpha}}
\newcommand{\A}{\mathbf{A}}
\def\a{\mathbf{a}}
\def\K{\mathbf{K}}
\newcommand{\B}{\mathbf{B}}
\def\b{\boldsymbol{\beta}}
\def\bhat{\hat{\boldsymbol{\beta}}}
\def\btilde{\tilde{\boldsymbol{\beta}}}
\def\tb{\tilde{\boldsymbol{\beta}}}
\def\bg{\boldsymbol{\beta_\gamma}}
\def\bgnot{\boldsymbol{\beta_{(-\gamma)}}}
\def\mub{\boldsymbol{\mu}}
\def\tmub{\tilde{\boldsymbol{\mu}}}
\def\muhat{\hat{\boldsymbol{\mu}}}
\def\t{\boldsymbol{\theta}}
\def\tk{\boldsymbol{\theta}_k}
\def\tj{\boldsymbol{\theta}_j}
\def\Mk{\boldsymbol{{\cal M}}_k}
\def\M{\boldsymbol{{\cal M}}}
\def\Mj{\boldsymbol{{\cal M}}_j}
\def\Mi{\boldsymbol{{\cal M}}_i}
\def\Mg{{\boldsymbol{{\cal M}_\gamma}}}
\def\Mnull{\boldsymbol{{\cal M}}_{N}}
\def\gMPM{\boldsymbol{\gamma}_{\text{MPM}}}
\def\gHPM{\boldsymbol{\gamma}_{\text{HPM}}}
\def\Mfull{\boldsymbol{{\cal M}}_{F}}
\def\tg{\boldsymbol{\theta}_{\boldsymbol{\gamma}}}
\def\g{\boldsymbol{\gamma}}
\def\eg{\boldsymbol{\eta}_{\boldsymbol{\gamma}}}
\def\G{\mathbf{G}}
\def\cM{\cal M}
\def\D{\Delta}
\def \shat{{\hat{\sigma}}^2}
\def\uv{\mathbf{u}}
\def\l {\lambda}
\def\d{\delta}
\def\Sigmab{\boldsymbol{\Sigma}}
\def\Lambdab{\boldsymbol{\Lambda}}
\def\lambdab{\boldsymbol{\lambda}}
\def\Mg{{\cal M}_\gamma}
\def\S{{\cal{S}}}
\def\qg{p_{\boldsymbol{\gamma}}}
\def\pg{p_{\boldsymbol{\gamma}}}
\def\t{\boldsymbol{\theta}}  
\def\T{\boldsymbol{\Theta}}  
\begin{document}
{\bf STA721 \hfill Homework 4}

\vspace{.5in}
\begin{enumerate}
\item Add 95\% prediction intervals to your plot from HW3 for the
  Prostate data using a different linetype and color.  Explain why the
  prediction intervals are wider than the confidence intervals for
  $\hat{\mub}$.  (See the function {\tt predict()} in \R.  Please
  label all axes with units and informative names, add a legend to
  explain the multiple lines, and a caption).   

\item Consider the linear model  model $\Y \sim \N(\mub, \sigma^2 \I_n)$
  with  $\mub = \one \beta_0 + \X \b$ and $\X$ a full rank matrix with
  rank $p$.  For a new observation $Y_*$ at $\x_*$ with $Y_* =
  \x_*^T\b + \epsilon_*$ and $\epsilon_*$ independent of $\eps$,  consider the
  predicted residual  $Y_* - \x^T_* \bhat$ where $\bhat$ is the MLE
  using data $\Y$.   
  \begin{enumerate}
  \item Find the distribution of the predicted residual $Y_* - \x^T_*
    \bhat$ given $\b$ and  $\sigma^2$. 
\item Show that the standardized predicted residual (center so that
  the mean is 1 and and scale (sd) is 
  1 with $\sigma^2$ replaced by the usual unbiased estimate
  $\hat{\sigma}^2 = \Y^T(\I-\P_{\X})\Y/ (n - p - 1)$ has a student $t$
  distribution.  What are the degrees of freedom?   
  \end{enumerate}
  \item   Consider the linear model $\Y = \X\b + \eps$ with $\E[\eps]
    = \zero_n$ and $\Cov(\eps) = \sigma^2 \I_n$ and with $\X$ of full
    column rank $(p+1)$.
  \begin{enumerate}
\item Consider estimation of $\b$ using quadratic loss $(\b -
  \a)^T(\b - \a)$ for some estimator $\a$.  Find the expected quadratic
  loss if we use the MLE $\bhat$ for $\a$. Simplify the expression
  as a function of the eigenvalues of $\X^T\X$.   What happens as the
  smallest eigenvalue goes to 0?
\item Consider estimation $\mub$'s at the observed data points
  $\X$.  Find the expected  quadratic loss   $\E[(\mub -
  \X\bhat)^T(\mub - \X\bhat)]$.  What happens as the  smallest eigen
  value of $\X^T\X$ goes to 0?
\item Consider predicting $\Y_*$'s at the observed data points $\X$
  where $\Y_*$ is independent of $\Y$.  Find the expected quadratic
  loss $\E[(\Y_* - \X\bhat)^T(\Y_* - \X\bhat)]$.  What happens as the
  smallest eigen value of $\X^T\X$ goes to 0?
\item Consider predicting $\Y_*$'s at new points $\X_*$ with
  $\E[\X_*^T\X_*] = \I_p$.  Find the expected quadratic loss
  $\E[(\Y_* - \X_*\bhat)^T(\Y_* - \X_*\bhat)]$.  What
  happens as the smallest eigen value of $\X^T\X$ goes to 0?  (If
  $E[\X_*^T\X_*] = \Sigmab > 0$ does that change the result)
\item Comment on the difference in estimation, prediction at observed
  data and prediction at new data as $\X$ becomes non-full rank.
  Which is the most stable?  Which is the least?
  \end{enumerate}
\end{enumerate}

\end{document}
