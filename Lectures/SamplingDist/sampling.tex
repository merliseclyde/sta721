\documentclass[handout]{beamer}
%\documentclass{beamer}
%\usepackage[dvips]{color}
\usepackage{graphicx}
\usepackage{amsmath,amssymb,array,comment,eucal}
\newcommand{\e}{\mathbf{e}}
\renewcommand{\P}{\mathbf{P}}
\newcommand{\F}{\mathbf{F}}
\newcommand{\R}{\textsf{R}}
\newcommand{\mat}[1] {\mathbf{#1}}
%\newcommand{\ind}{\mathrel{\mathop{\sim}\limits^{\mathit{ind}}}}
%\newcommand{\iid}{\mathrel{\mathop{\sim}\limits^{\mathit{iid}}}}
\newcommand{\E}{\textsf{E}}
\newcommand{\SE}{\textsf{SE}}
\newcommand{\SSE}{\textsf{SSE}}
\newcommand{\RSS}{\textsf{RSS}}
\newcommand{\FSS}{\textsf{FSS}}
\renewcommand{\SS}{\textsf{SS}}
\newcommand{\MSE}{\textsf{MSE}}
\newcommand{\SSR}{\textsf{SSR}}
\newcommand{\Be}{\textsf{Beta}}
\newcommand{\St}{\textsf{St}}
%\newcommand{\C}{\textsf{C}}
\newcommand{\GDP}{\textsf{GDP}}
\newcommand{\NcSt}{\textsf{NcSt}}
\newcommand{\Bin}{\textsf{Bin}}
\newcommand{\NB}{\textsf{NegBin}}
\renewcommand{\NG}{\textsf{NG}}
\newcommand{\N}{\textsf{N}}
\newcommand{\Ber}{\textsf{Ber}}
\newcommand{\Poi}{\text{Poi}}
\newcommand{\Gam}{\textsf{Gamma}}
\newcommand{\BB}{\textsf{BB}}
\newcommand{\Gm}{\textsf{G}}
\newcommand{\Un}{\textsf{Unif}}
\newcommand{\Ex}{\textsf{Exp}}
\newcommand{\DE}{\textsf{DE}}
\newcommand{\tr}{\textsf{tr}}
\newcommand{\cF}{{\cal{F}}}
\newcommand{\cL}{{\cal{L}}}
\newcommand{\cI}{{\cal{I}}}
\newcommand{\cB}{{\cal{B}}}
\newcommand{\cP}{{\cal{P}}}
\newcommand{\bbR}{\mathbb{R}}
\newcommand{\bbN}{\mathbb{N}}
\newcommand{\pperp}{\mathrel{{\rlap{$\,\perp$}\perp\,\,}}}
\newcommand{\OFP}{(\Omega,\cF, \P)}
\newcommand{\eps}{\boldsymbol{\epsilon}}
\newcommand{\1}{\mathbf{1}_n}
\newcommand{\gap}{\vspace{8mm}}
\newcommand{\ind}{\mathrel{\mathop{\sim}\limits^{\rm ind}}}
\newcommand{\simiid}{\ensuremath{\mathrel{\mathop{\sim}\limits^{\rm
iid}}}}
\newcommand{\eqindis}{\ensuremath{\mathrel{\mathop{=}\limits^{\rm D}}}}
\newcommand{\iid}{\textit{i.i.d.}}
\newcommand{\SSZ}{S_{zz}}
\newcommand{\SZW}{S_{zw}}
\newcommand{\Var}{\textsf{Var}}
\newcommand{\corr}{\textsf{corr}}
\newcommand{\diag}{\textsf{diag}}
\newcommand{\var}{\textsf{var}}
\newcommand{\Cov}{\textsf{Cov}}
\newcommand{\Sam}{{\cal S}}
\def\H{\mathbf{H}}
\newcommand{\I}{\mathbf{I}}
\newcommand{\Y}{\mathbf{Y}}
\newcommand{\tY}{\tilde{\mathbf{Y}}}
\newcommand{\Yhat}{\hat{\mathbf{Y}}}
\newcommand{\Yobs}{\mathbf{Y}_{{\cal S}}}
\newcommand{\barYobs}{\bar{Y}_{{\cal S}}}
\newcommand{\barYmiss}{\bar{Y}_{{\cal S}^c}}
\def\bv{\mathbf{b}}
\def\X{\mathbf{X}}
\def\tX{\tilde{\mathbf{X}}}
\def\x{\mathbf{x}}
\def\xbar{\bar{\mathbf{x}}}
\def\Xbar{\bar{\mathbf{X}}}
\def\Xg{\mathbf{X}_{\boldsymbol{\gamma}}}
\def\Ybar{\bar{\Y}}
\def\ybar{\bar{y}}
\def\y{\mathbf{y}}
\def\Yf{\mathbf{Y_f}}
\def\W{\mathbf{W}}
\def\L{\mathbf{L}}
\def\w{\mathbf{w}}
\def\U{\mathbf{U}}
\def\V{\mathbf{V}}
\def\Q{\mathbf{Q}}
\def\Z{\mathbf{Z}}
\def\z{\mathbf{z}}
\def\v{\mathbf{v}}
\def\u{\mathbf{u}}

\def\zero{\mathbf{0}}
\def\one{\mathbf{1}}
\newcommand{\taub}{\boldsymbol{\tau}}
\newcommand{\betav}{\boldsymbol{\beta}}
\newcommand{\alphav}{\boldsymbol{\alpha}}
\newcommand{\A}{\mathbf{A}}
\def\a{\mathbf{a}}
\def\K{\mathbf{K}}
\newcommand{\B}{\mathbf{B}}
\def\b{\boldsymbol{\beta}}
\def\bhat{\hat{\boldsymbol{\beta}}}
\def\btilde{\tilde{\boldsymbol{\beta}}}
\def\tb{\tilde{\boldsymbol{\beta}}}
\def\bg{\boldsymbol{\beta_\gamma}}
\def\bgnot{\boldsymbol{\beta_{(-\gamma)}}}
\def\mub{\boldsymbol{\mu}}
\def\tmub{\tilde{\boldsymbol{\mu}}}
\def\muhat{\hat{\boldsymbol{\mu}}}
\def\t{\boldsymbol{\theta}}
\def\tk{\boldsymbol{\theta}_k}
\def\tj{\boldsymbol{\theta}_j}
\def\Mk{\boldsymbol{{\cal M}}_k}
\def\M{\boldsymbol{{\cal M}}}
\def\Mj{\boldsymbol{{\cal M}}_j}
\def\Mi{\boldsymbol{{\cal M}}_i}
\def\Mg{{\boldsymbol{{\cal M}_\gamma}}}
\def\Mnull{\boldsymbol{{\cal M}}_{N}}
\def\gMPM{\boldsymbol{\gamma}_{\text{MPM}}}
\def\gHPM{\boldsymbol{\gamma}_{\text{HPM}}}
\def\Mfull{\boldsymbol{{\cal M}}_{F}}
\def\tg{\boldsymbol{\theta}_{\boldsymbol{\gamma}}}
\def\g{\boldsymbol{\gamma}}
\def\eg{\boldsymbol{\eta}_{\boldsymbol{\gamma}}}
\def\G{\mathbf{G}}
\def\cM{\cal M}
\def\D{\Delta}
\def \shat{{\hat{\sigma}}^2}
\def\uv{\mathbf{u}}
\def\l {\lambda}
\def\d{\delta}
\def\Sigmab{\boldsymbol{\Sigma}}
\def\Lambdab{\boldsymbol{\Lambda}}
\def\lambdab{\boldsymbol{\lambda}}
\def\Mg{{\cal M}_\gamma}
\def\S{{\cal{S}}}
\def\qg{p_{\boldsymbol{\gamma}}}
\def\pg{p_{\boldsymbol{\gamma}}}
\def\t{\boldsymbol{\theta}}  
\def\T{\boldsymbol{\Theta}}  
\usepackage{verbatim}

\usetheme{Warsaw}
\title{Sampling Distributions}
\subtitle{Merlise Clyde}
\author{STA721 Linear Models}
\institute{Duke University}
\date{September 3, 2015}
\logo{duke.eps}

\begin{document}
\maketitle

\begin{frame}\frametitle{Outline}
Topics 
  \begin{itemize}
  \item Normal Theory
  \item Chi-squared Distributions
  \item Student $t$ Distributions
  \end{itemize}

\vspace{24pt}
Readings:  Christensen Apendix C, Chapter 1-2 
\end{frame}
%\section{Models}

\begin{frame}[fragile]
  \frametitle{Prostate Example}
  \begin{small}
\begin{verbatim}
> library(lasso2); data(Prostate)    # n = 97, 9 variables
> summary(lm(lpsa ~ ., data=Prostate))
Coefficients:
             Estimate Std. Error t value Pr(>|t|)    
(Intercept)  0.669399   1.296381   0.516  0.60690    
lcavol       0.587023   0.087920   6.677 2.11e-09 ***
lweight      0.454461   0.170012   2.673  0.00896 ** 
age         -0.019637   0.011173  -1.758  0.08229 .  
lbph         0.107054   0.058449   1.832  0.07040 .  
svi          0.766156   0.244309   3.136  0.00233 ** 
lcp         -0.105474   0.091013  -1.159  0.24964    
gleason      0.045136   0.157464   0.287  0.77506    
pgg45        0.004525   0.004421   1.024  0.30885    
---
Signif. codes:  0 '***' 0.001 '**' 0.01 '*' 0.05 '.' 0.1 ' ' 1

Residual standard error: 0.7084 on 88 degrees of freedom
Multiple R-squared:  0.6548,	Adjusted R-squared:  0.6234 
F-statistic: 20.86 on 8 and 88 DF,  p-value: < 2.2e-16
\end{verbatim}
    
\end{small}

\end{frame}

\begin{frame}
  \frametitle{Summary of Distributions}

Models:  Full  $\Y = \X\b + \eps$ 

Assume $\X$ is full rank with the first column of ones $\one_n$ and $p$ additional predictors $r(\X) = p +1$

   $$\hat{\b} \mid \sigma^2 \sim \N(\b, \sigma^2 (\X^T\X)^{-1})$$
   $$\frac{\RSS }{\sigma^2} \sim \chi^2_{n - r(\X)}$$
   $$\frac{\hat{\beta}_j - \beta_j}{\SE(\hat{\beta}_j)} \sim t_{n - r(\X)}$$
where  $\SE(\hat{\beta})$  is the square root of the  $j$th diagonal element of $\hat{\sigma}^2 (\X^T\X)^{-1}$   and $\hat{\sigma}^2$ is the unbiased estimate of $\sigma^2$
\end{frame}

\section{Sampling Distribution}



\begin{frame}
  \frametitle{Sampling Distribution of $\b$}
If  $\Y \sim \N(\X\b, \sigma^2 \I_n)$

Then $\hat{\b} \sim \N(\b, \sigma^2 (\X^T\X)^{-1})$
\vfill

\end{frame}


\begin{frame} \frametitle{Unknown $\sigma^2$}

  \begin{align*}
    \hat{\beta}_j \mid \beta_j, \sigma^2 & \sim \N(\beta, { \sigma^2}
                                           [(\X^T\X)^{-1}]_{jj})
  \end{align*}  \pause

What happens if we substitute $\hat{\sigma}^2 = \e^t\e/(n-r(\X))$ in the above? \pause

$$
\frac{(\hat{\beta}_j - \beta_j)/ \sigma \sqrt{ [(\X^T\X)^{-1}]_{jj}}}
{\sqrt{\e^T\e/ (\sigma^2 (n - r(\X))} } \eqindis
\frac{N(0,1)}{\sqrt{\chi_{n-r(\X)}^2/(n - r(\X)}} \sim t(n - r(\X), 0 ,1)
$$

Need to show that $\e^T\e /\sigma^2$ has a $\chi^2$ distribution and
is independent of the numerator!
\end{frame}

\begin{frame}
  \frametitle{Central Student $t$ Distribution}
  \begin{definition}
    Let $Z \sim \N(0, 1)$ and $S \sim \chi^2_p$ with $Z$ and $S$
    independent, \pause then
 $$ W = \frac{Z} {\sqrt{S/p}}$$
has a (central) Student $t$ distribution with $p$ degrees of freedom
  \end{definition}
\pause
 
See Casella \& Berger or DeGroot \& Schervish for derivation - nice change of variables and marginalization problem!
\end{frame}






\begin{frame}
  \frametitle{Chi-Squared Distribution}
  \begin{Definition}
    If $Z \sim \N(0,1)$ then $Z^2 \sim \chi^2_1$ (A Chi-squared
    distribution with one degree of freedom) \pause
    \begin{itemize}
    \item Density
$$
f(x) = \frac{1}{\Gamma(1/2)} (1/2)^{-1/2} x^{1/2 - 1} e^{-x/2} \qquad x
> 0
$$ \pause
  \item  Characteristic Function
$$
\E[e^{itZ^2}] = \varphi(t) = (1 - 2 i t)^{-1/2}
$$
  \end{itemize}

\end{Definition}
\end{frame}

\begin{frame}
  \frametitle{Chi-Squared Distribution with $p$ Degrees of Freedom}
If $Z_j \simiid \N(0,1)$ $j = 1, \ldots p$ then $X \equiv \Z^T\Z = \sum_j^p
Z_j^2 \sim \chi^2_p$  \pause

\begin{block}{Characteristic Function}
\begin{eqnarray*}
  \varphi_{X}(t) & = & \E[e^{it \sum_j^p
Z_j^2}] \pause \\
& = & \prod_{j=1}^p \E[e^{it Z_j^2 }] \pause \\
& = &  \prod_{j=1}^p (1 - 2 i t)^{-1/2} \pause \\
& = & (1 - 2 i t)^{-p/2} \pause
\end{eqnarray*}
A  Gamma distribution with shape $p/2$ and rate $1/2$, $G(p/2, 1/2)$ 

$$
f(x) = \frac{1}{\Gamma(p/2)} (1/2)^{-p/2} x^{p/2 - 1} e^{-x/2} \qquad x
> 0
$$
 \end{block}
\end{frame}
\begin{frame}
  \frametitle{Quadratic Forms}
  \begin{theorem}
  Let  $\Y \sim  \N(\mub, \sigma^2 \I_n)$ with $\mub \in C(\X)$ then if $\Q$ is
  a rank $k$ orthogonal  projection on to $C(\X)^{\perp}$,
$(\Y^T \Q \Y)/\sigma^2 \sim \chi^2_k$
  \end{theorem}
  \begin{proof}
    For an orthogonal projection  $\Q = \U \Lambdab \U^T 
    = \U_k \U_k^T$ where $C(\Q) = C(\U_k)$ and $\U_k^T\U_k = \I_k$
    (Spectral Theorem) \pause
    \begin{eqnarray*}
\Y^T\Q \Y  &=& \Y^T\U_k \U_k^T \Y      \pause \\
\Z & =  &\U_k^T \Y/ \sigma  \sim \N(\U_k^T\mub, \U_k^T\U_k) \pause \\
\Z & \sim & \N(\zero, \I_k) \pause \\
\Z^T\Z & \sim & \chi^2_k \pause 
    \end{eqnarray*}
Since $U^T\Y /\sigma \eqindis \Z$, $\frac{\Y^T\Q \Y}{\sigma^2} \sim
\chi^2_k$


  \end{proof}
\end{frame}

\begin{frame} \frametitle{Residual Sum of Squares Example}
 
\begin{block}{Sum of Squares Error  (SSE)}
Let $\Y \sim \N(\mub, \sigma^2 \I_n)$ with $\mub \in C(\X)$.


Because $\mub \in C(\X)$, $\I - \P_{\X}$ is a projection on
$C(\X)^{\perp}$ \pause $$\frac{\e^T\e}{\sigma^2} = \Y^T\frac{(\I_n - \P_\X)}\sigma^2 \Y  \sim
 \chi^2_{n - r(\X)}$$ 
\end{block}
\end{frame}

\begin{frame}
  \frametitle{ Estimated Coefficients and Residuals are Independent}
If  $\Y \sim \N(\X\b, \sigma^2 \I_n)$

Then $\Cov(\hat{\b}, \e) = \zero$ which implies independence 
\vfill

Functions of independent random variables are independent
(show characteristic functions or densities factor)
\end{frame}

\begin{frame}\frametitle{Putting it all together}
$\hat{\b} \sim \N(\b, \sigma^2 (\X^T\X)^{-1})$
  \begin{itemize}
  \item $(\hat{\beta}_j - \beta_j)/ \sigma [(\X^T\X)^{-1}]_{jj} \sim
    \N(0,1)$
\item $\e^T\e/ \sigma^2 \sim \chi^2_{n - r(\X)}$
\item $\hat{\beta}$ and $\e$ are independent
$$ \frac{(\hat{\beta}_j - \beta_j)/ \sigma [(\X^T\X)^{-1}]_{jj}}
{\sqrt{\e^T\e/ ( \sigma^2 (n - r(\X)))}} \sim t(n - r(\X), 0, 1)$$
  \end{itemize}
\end{frame}

\begin{frame}
  \frametitle{Inference}
\vspace{-.5in}

  \begin{itemize}
  \item 
   95\% Confidence interval:  $\hat{\beta}_j \pm t_{\alpha/2}
   \SE(\hat{\beta}_j)$  \pause use {\tt qt(a, df)} for $t_a$ quantile
\item derive from pivotal quantity $t = (\hat{\beta}_j -
  \beta_j)/\SE(\hat{\beta}_j)$ where 
 $P(t \in (t_{\alpha/2}, t_{1 - \alpha/2}))  = 1 -\alpha$
  \end{itemize}

\vfill

\end{frame}
\begin{frame}\frametitle{Prostate Example}
{\tt xtable(confint(prostate.lm))}  from library(MASS) and library(xtable)
  \begin{table}[ht]
\centering
\begin{tabular}{rrr}
  \hline
 & 2.5 \% & 97.5 \% \\ 
  \hline
(Intercept) & -1.91 & 3.25 \\ 
  lcavol & 0.41 & 0.76 \\ 
  lweight & 0.12 & 0.79 \\ 
  age & -0.04 & 0.00 \\ 
  lbph & -0.01 & 0.22 \\ 
  svi & 0.28 & 1.25 \\ 
  lcp & -0.29 & 0.08 \\ 
  gleason & -0.27 & 0.36 \\ 
  pgg45 & -0.00 & 0.01 \\ 
   \hline
\end{tabular}
\end{table}
\end{frame}
\begin{frame} \frametitle{interpretation}
  \begin{itemize}
  \item 
  For a ``1'' unit increase in $\X_j$, expect $\Y$ to increase by $\hat{\beta}_j \pm t_{\alpha/2}
   \SE(\hat{\beta}_j)$ 
\item for log transforms
$$\Y = \exp(\X\b + \eps) = \prod \exp(\X_j \beta_j) \exp(\eps)$$
\item if $\X = \log(\W_j)$ then look at 2-fold or \%
  increases in $\W$ to look at multiplicative increase in median of $\Y$
\item if{\tt  cavol} increases by 10\%  then we expect {\tt PSA} to increase
  by $1.10^{(CI)}$  = $( 1.0398 \%, 1.0751 \%)$ or by 3.98 to 7.51 percent
  \end{itemize}
For a 10\% increase in cancer volume, we are 95\% confident  that the PSA levels
will increase by approximately 4 to 7.5 percent.
\end{frame}
\begin{frame}\frametitle{Derivation}
  
\end{frame}
\end{document}




