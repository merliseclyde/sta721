\documentclass[12pt]{article}
\usepackage{fullpage,amssymb,amsmath, url, hyperref}
\pagestyle{empty}
\newcommand{\e}{\mathbf{e}}
\renewcommand{\P}{\mathbf{P}}
\newcommand{\F}{\mathbf{F}}
\newcommand{\R}{\textsf{R}}
\newcommand{\mat}[1] {\mathbf{#1}}
%\newcommand{\ind}{\mathrel{\mathop{\sim}\limits^{\mathit{ind}}}}
%\newcommand{\iid}{\mathrel{\mathop{\sim}\limits^{\mathit{iid}}}}
\newcommand{\E}{\textsf{E}}
\newcommand{\SE}{\textsf{SE}}
\newcommand{\SSE}{\textsf{SSE}}
\newcommand{\RSS}{\textsf{RSS}}
\newcommand{\FSS}{\textsf{FSS}}
\renewcommand{\SS}{\textsf{SS}}
\newcommand{\MSE}{\textsf{MSE}}
\newcommand{\SSR}{\textsf{SSR}}
\newcommand{\Be}{\textsf{Beta}}
\newcommand{\St}{\textsf{St}}
%\newcommand{\C}{\textsf{C}}
\newcommand{\GDP}{\textsf{GDP}}
\newcommand{\NcSt}{\textsf{NcSt}}
\newcommand{\Bin}{\textsf{Bin}}
\newcommand{\NB}{\textsf{NegBin}}
\renewcommand{\NG}{\textsf{NG}}
\newcommand{\N}{\textsf{N}}
\newcommand{\Ber}{\textsf{Ber}}
\newcommand{\Poi}{\text{Poi}}
\newcommand{\Gam}{\textsf{Gamma}}
\newcommand{\BB}{\textsf{BB}}
\newcommand{\Gm}{\textsf{G}}
\newcommand{\Un}{\textsf{Unif}}
\newcommand{\Ex}{\textsf{Exp}}
\newcommand{\DE}{\textsf{DE}}
\newcommand{\tr}{\textsf{tr}}
\newcommand{\cF}{{\cal{F}}}
\newcommand{\cL}{{\cal{L}}}
\newcommand{\cI}{{\cal{I}}}
\newcommand{\cB}{{\cal{B}}}
\newcommand{\cP}{{\cal{P}}}
\newcommand{\bbR}{\mathbb{R}}
\newcommand{\bbN}{\mathbb{N}}
\newcommand{\pperp}{\mathrel{{\rlap{$\,\perp$}\perp\,\,}}}
\newcommand{\OFP}{(\Omega,\cF, \P)}
\newcommand{\eps}{\boldsymbol{\epsilon}}
\newcommand{\1}{\mathbf{1}_n}
\newcommand{\gap}{\vspace{8mm}}
\newcommand{\ind}{\mathrel{\mathop{\sim}\limits^{\rm ind}}}
\newcommand{\simiid}{\ensuremath{\mathrel{\mathop{\sim}\limits^{\rm
iid}}}}
\newcommand{\eqindis}{\ensuremath{\mathrel{\mathop{=}\limits^{\rm D}}}}
\newcommand{\iid}{\textit{i.i.d.}}
\newcommand{\SSZ}{S_{zz}}
\newcommand{\SZW}{S_{zw}}
\newcommand{\Var}{\textsf{Var}}
\newcommand{\corr}{\textsf{corr}}
\newcommand{\diag}{\textsf{diag}}
\newcommand{\var}{\textsf{var}}
\newcommand{\Cov}{\textsf{Cov}}
\newcommand{\Sam}{{\cal S}}
\def\H{\mathbf{H}}
\newcommand{\I}{\mathbf{I}}
\newcommand{\Y}{\mathbf{Y}}
\newcommand{\tY}{\tilde{\mathbf{Y}}}
\newcommand{\Yhat}{\hat{\mathbf{Y}}}
\newcommand{\Yobs}{\mathbf{Y}_{{\cal S}}}
\newcommand{\barYobs}{\bar{Y}_{{\cal S}}}
\newcommand{\barYmiss}{\bar{Y}_{{\cal S}^c}}
\def\bv{\mathbf{b}}
\def\X{\mathbf{X}}
\def\tX{\tilde{\mathbf{X}}}
\def\x{\mathbf{x}}
\def\xbar{\bar{\mathbf{x}}}
\def\Xbar{\bar{\mathbf{X}}}
\def\Xg{\mathbf{X}_{\boldsymbol{\gamma}}}
\def\Ybar{\bar{\Y}}
\def\ybar{\bar{y}}
\def\y{\mathbf{y}}
\def\Yf{\mathbf{Y_f}}
\def\W{\mathbf{W}}
\def\L{\mathbf{L}}
\def\w{\mathbf{w}}
\def\U{\mathbf{U}}
\def\V{\mathbf{V}}
\def\Q{\mathbf{Q}}
\def\Z{\mathbf{Z}}
\def\z{\mathbf{z}}
\def\v{\mathbf{v}}
\def\u{\mathbf{u}}

\def\zero{\mathbf{0}}
\def\one{\mathbf{1}}
\newcommand{\taub}{\boldsymbol{\tau}}
\newcommand{\betav}{\boldsymbol{\beta}}
\newcommand{\alphav}{\boldsymbol{\alpha}}
\newcommand{\A}{\mathbf{A}}
\def\a{\mathbf{a}}
\def\K{\mathbf{K}}
\newcommand{\B}{\mathbf{B}}
\def\b{\boldsymbol{\beta}}
\def\bhat{\hat{\boldsymbol{\beta}}}
\def\btilde{\tilde{\boldsymbol{\beta}}}
\def\tb{\tilde{\boldsymbol{\beta}}}
\def\bg{\boldsymbol{\beta_\gamma}}
\def\bgnot{\boldsymbol{\beta_{(-\gamma)}}}
\def\mub{\boldsymbol{\mu}}
\def\tmub{\tilde{\boldsymbol{\mu}}}
\def\muhat{\hat{\boldsymbol{\mu}}}
\def\t{\boldsymbol{\theta}}
\def\tk{\boldsymbol{\theta}_k}
\def\tj{\boldsymbol{\theta}_j}
\def\Mk{\boldsymbol{{\cal M}}_k}
\def\M{\boldsymbol{{\cal M}}}
\def\Mj{\boldsymbol{{\cal M}}_j}
\def\Mi{\boldsymbol{{\cal M}}_i}
\def\Mg{{\boldsymbol{{\cal M}_\gamma}}}
\def\Mnull{\boldsymbol{{\cal M}}_{N}}
\def\gMPM{\boldsymbol{\gamma}_{\text{MPM}}}
\def\gHPM{\boldsymbol{\gamma}_{\text{HPM}}}
\def\Mfull{\boldsymbol{{\cal M}}_{F}}
\def\tg{\boldsymbol{\theta}_{\boldsymbol{\gamma}}}
\def\g{\boldsymbol{\gamma}}
\def\eg{\boldsymbol{\eta}_{\boldsymbol{\gamma}}}
\def\G{\mathbf{G}}
\def\cM{\cal M}
\def\D{\Delta}
\def \shat{{\hat{\sigma}}^2}
\def\uv{\mathbf{u}}
\def\l {\lambda}
\def\d{\delta}
\def\Sigmab{\boldsymbol{\Sigma}}
\def\Lambdab{\boldsymbol{\Lambda}}
\def\lambdab{\boldsymbol{\lambda}}
\def\Mg{{\cal M}_\gamma}
\def\S{{\cal{S}}}
\def\qg{p_{\boldsymbol{\gamma}}}
\def\pg{p_{\boldsymbol{\gamma}}}
\def\t{\boldsymbol{\theta}}  
\def\T{\boldsymbol{\Theta}}  
\begin{document}
{\bf STA721}
\vspace{.1in}
\begin{center}
{\large \bf TakeHome} \\
\end{center}

The paper by
\href{http://search.proquest.com.proxy.lib.duke.edu/docview/220285502/781DFACF0348441BPQ/1?accountid=10598}{Berry
  et al (2009) Clinical Trials 6:28-41}
presents an approach to Meta-Analysis using Bayesian Model Averaging.
The data consist of summaries from 22 studies (a row in the data file)
that explore the relationship between  vitamin E doses and all-cause
mortality.  The paper  focuses on 4 functions for how vitamine E dose
enters the mean:
1) no effect; 2) a quadratic-linear spline; 3) a linear effect; and 4)
a quadratic does effect.  While several other covariates are
entertained in addition to the dose of vitamen E, their main results are
based on the 4 models in vitamin E with uniform prior probabilities
over the 4 models.

This BMA approach has been critiqued by 
\href{http://search.proquest.com.proxy.lib.duke.edu/docview/220281455/fulltextPDF/4857C29367D34842PQ/2?accountid=10598}{
  Greenland (2009) Clinical Trials 6:50-51}
who argues that ``point null hypotheses'' of no effect should not be
considered and that results are driven by the prior probability (0.25)
placed on the null model.

The response for the meta-analysis is the empirical log-odds ratio:

$$ d_i = log\left( \frac{\hat{p}(D=1 \mid T_i=E_i)/(1 - \hat{ p}(D=1 \mid T_i=E_i))}
{\hat{p}(D=1 \mid T_i=C_i)/(1 - \hat{p}(D=1 \mid T_i=C_i))}\right)
$$
where $\hat{p}(D=1 \mid T_i=E_i) = n_{i D E}/n_{i E}$, $n_{i D E}$ is
the number of deaths in the treated group, $n_{i E}$ is the total
number in the treated group, $\hat{p}(D=1 \mid T_i=U_i) = n_{i D U}/n_{i U}$, $n_{i D U}$ is
the number of deaths in the untreated group, $n_{i U}$ is the total
number in the untreated group (controls)

Asymptotically
\begin{align}
d_i  & \ind \N(\mu_i, \sigma^2 s_i^2)  \\
s_i^2 & = \frac{1}{n_{i D E}} + \frac{1}{n_{i E} - n_{i D E}} +
 \frac{1}{n_{i D U}} + \frac{1}{n_{i U} - n_{i D U}} 
\end{align}
where if the model is correct, $\sigma^2 = 1$ (i.e. no over or under
dispersion relative to the binomial model).   This would require
Weighted Regression to find MLEs or the likelihood, $\mub \in C(\X)$.

Read the paper and discussion.  The objective of this TakeHomw is to
reanalyze the Vitamin E data from a Bayesian perspective using both
model averaging and from a single ``rich'' model that captures
important uncertainties described in the paper and comment.  Create
credible/confidence intervals (ideally graphs) to show the effect of
vitamin E on all-cause mortality after adjusting for (any/all)
covariates.  

Your write-up should provide: a (1) (brief) introduction to the
problem; (2) the methods/models that you have used in enough detail
that someone with the statistical background of this course could
reproduce your model without having the code (i.e models, priors,
discussion of hyperparameters; (3) results (discuss
similarities/differences of frequentist intervals, BMA and other
models, sensitivity to error/model/prior assumptions, does vitamin E
supplementation appear to be harmful or beneficial (at all or at
specific doses); and (4) overall discussion/summary and
recommendations.  Do you think that BMA is appropriate here or do you
recommend a different model?  Quantitatively how sensetive are results
to modelling assumptions at all stages?


Prepare a report using R/Rmarkdown or related of up to two pages.
Where possible illustrate main conclusions with figures (up to 2 in
the main body) 

You do not need to describe all of the exploratory work that you do or
wrong turns, but include the most important points in the body.  The
JAGS model  or other R code may be included in a supplemental
appendix.  

A few things to consider (but you do not need to address point-by-point): 

\begin{itemize}
\item To start, can you replicate any of the figures in the paper?
\item Are there any studies that are outliers or violations of the
  normality assumption?  What do you think is
  the best way to address this if this is an issue?
\item Is there evidence of over/under dispersion?  How sensitive is
  this to the prior?  Is a ``default'' reference prior approriate here?
\item Which assumptions affect the results the most?  (using weights,
  model for likelihood, form for how vit E enters the mean function,
  priors on coefficients, priors on models)?
\item What is the justification for your prior(s)?  If you simulate data
  based on generating from the prior/model do the outcomes seem
  plausible?  (what does the prior say about probabilities instead of
  log-odds ratios?)
\item How could validate the model?  How would you create
  validation/test data?   Is out-of-sample prediction relevant to
  effect quantification?
\item for energetic students,  could you fit this with logistic
  regression and if so do conclusions change?
\item In writing-up results, be sure to not only address questions
  qualitative, but quantify estimates with point estimates and
  uncertainty intervals, with interpretations.
\end{itemize}
\end{document}