\documentclass[12pt]{article}
\usepackage{fullpage}
\usepackage{amsmath,amssymb,array,comment,eucal}
\pagestyle{empty}
\newcommand{\e}{\mathbf{e}}
\renewcommand{\P}{\mathbf{P}}
\newcommand{\F}{\mathbf{F}}
\newcommand{\R}{\textsf{R}}
\newcommand{\mat}[1] {\mathbf{#1}}
%\newcommand{\ind}{\mathrel{\mathop{\sim}\limits^{\mathit{ind}}}}
%\newcommand{\iid}{\mathrel{\mathop{\sim}\limits^{\mathit{iid}}}}
\newcommand{\E}{\textsf{E}}
\newcommand{\SE}{\textsf{SE}}
\newcommand{\SSE}{\textsf{SSE}}
\newcommand{\RSS}{\textsf{RSS}}
\newcommand{\FSS}{\textsf{FSS}}
\renewcommand{\SS}{\textsf{SS}}
\newcommand{\MSE}{\textsf{MSE}}
\newcommand{\SSR}{\textsf{SSR}}
\newcommand{\Be}{\textsf{Beta}}
\newcommand{\St}{\textsf{St}}
%\newcommand{\C}{\textsf{C}}
\newcommand{\GDP}{\textsf{GDP}}
\newcommand{\NcSt}{\textsf{NcSt}}
\newcommand{\Bin}{\textsf{Bin}}
\newcommand{\NB}{\textsf{NegBin}}
\renewcommand{\NG}{\textsf{NG}}
\newcommand{\N}{\textsf{N}}
\newcommand{\Ber}{\textsf{Ber}}
\newcommand{\Poi}{\text{Poi}}
\newcommand{\Gam}{\textsf{Gamma}}
\newcommand{\BB}{\textsf{BB}}
\newcommand{\Gm}{\textsf{G}}
\newcommand{\Un}{\textsf{Unif}}
\newcommand{\Ex}{\textsf{Exp}}
\newcommand{\DE}{\textsf{DE}}
\newcommand{\tr}{\textsf{tr}}
\newcommand{\cF}{{\cal{F}}}
\newcommand{\cL}{{\cal{L}}}
\newcommand{\cI}{{\cal{I}}}
\newcommand{\cB}{{\cal{B}}}
\newcommand{\cP}{{\cal{P}}}
\newcommand{\bbR}{\mathbb{R}}
\newcommand{\bbN}{\mathbb{N}}
\newcommand{\pperp}{\mathrel{{\rlap{$\,\perp$}\perp\,\,}}}
\newcommand{\OFP}{(\Omega,\cF, \P)}
\newcommand{\eps}{\boldsymbol{\epsilon}}
\newcommand{\1}{\mathbf{1}_n}
\newcommand{\gap}{\vspace{8mm}}
\newcommand{\ind}{\mathrel{\mathop{\sim}\limits^{\rm ind}}}
\newcommand{\simiid}{\ensuremath{\mathrel{\mathop{\sim}\limits^{\rm
iid}}}}
\newcommand{\eqindis}{\ensuremath{\mathrel{\mathop{=}\limits^{\rm D}}}}
\newcommand{\iid}{\textit{i.i.d.}}
\newcommand{\SSZ}{S_{zz}}
\newcommand{\SZW}{S_{zw}}
\newcommand{\Var}{\textsf{Var}}
\newcommand{\corr}{\textsf{corr}}
\newcommand{\diag}{\textsf{diag}}
\newcommand{\var}{\textsf{var}}
\newcommand{\Cov}{\textsf{Cov}}
\newcommand{\Sam}{{\cal S}}
\def\H{\mathbf{H}}
\newcommand{\I}{\mathbf{I}}
\newcommand{\Y}{\mathbf{Y}}
\newcommand{\tY}{\tilde{\mathbf{Y}}}
\newcommand{\Yhat}{\hat{\mathbf{Y}}}
\newcommand{\Yobs}{\mathbf{Y}_{{\cal S}}}
\newcommand{\barYobs}{\bar{Y}_{{\cal S}}}
\newcommand{\barYmiss}{\bar{Y}_{{\cal S}^c}}
\def\bv{\mathbf{b}}
\def\X{\mathbf{X}}
\def\tX{\tilde{\mathbf{X}}}
\def\x{\mathbf{x}}
\def\xbar{\bar{\mathbf{x}}}
\def\Xbar{\bar{\mathbf{X}}}
\def\Xg{\mathbf{X}_{\boldsymbol{\gamma}}}
\def\Ybar{\bar{\Y}}
\def\ybar{\bar{y}}
\def\y{\mathbf{y}}
\def\Yf{\mathbf{Y_f}}
\def\W{\mathbf{W}}
\def\L{\mathbf{L}}
\def\w{\mathbf{w}}
\def\U{\mathbf{U}}
\def\V{\mathbf{V}}
\def\Q{\mathbf{Q}}
\def\Z{\mathbf{Z}}
\def\z{\mathbf{z}}
\def\v{\mathbf{v}}
\def\u{\mathbf{u}}

\def\zero{\mathbf{0}}
\def\one{\mathbf{1}}
\newcommand{\taub}{\boldsymbol{\tau}}
\newcommand{\betav}{\boldsymbol{\beta}}
\newcommand{\alphav}{\boldsymbol{\alpha}}
\newcommand{\A}{\mathbf{A}}
\def\a{\mathbf{a}}
\def\K{\mathbf{K}}
\newcommand{\B}{\mathbf{B}}
\def\b{\boldsymbol{\beta}}
\def\bhat{\hat{\boldsymbol{\beta}}}
\def\btilde{\tilde{\boldsymbol{\beta}}}
\def\tb{\tilde{\boldsymbol{\beta}}}
\def\bg{\boldsymbol{\beta_\gamma}}
\def\bgnot{\boldsymbol{\beta_{(-\gamma)}}}
\def\mub{\boldsymbol{\mu}}
\def\tmub{\tilde{\boldsymbol{\mu}}}
\def\muhat{\hat{\boldsymbol{\mu}}}
\def\t{\boldsymbol{\theta}}
\def\tk{\boldsymbol{\theta}_k}
\def\tj{\boldsymbol{\theta}_j}
\def\Mk{\boldsymbol{{\cal M}}_k}
\def\M{\boldsymbol{{\cal M}}}
\def\Mj{\boldsymbol{{\cal M}}_j}
\def\Mi{\boldsymbol{{\cal M}}_i}
\def\Mg{{\boldsymbol{{\cal M}_\gamma}}}
\def\Mnull{\boldsymbol{{\cal M}}_{N}}
\def\gMPM{\boldsymbol{\gamma}_{\text{MPM}}}
\def\gHPM{\boldsymbol{\gamma}_{\text{HPM}}}
\def\Mfull{\boldsymbol{{\cal M}}_{F}}
\def\tg{\boldsymbol{\theta}_{\boldsymbol{\gamma}}}
\def\g{\boldsymbol{\gamma}}
\def\eg{\boldsymbol{\eta}_{\boldsymbol{\gamma}}}
\def\G{\mathbf{G}}
\def\cM{\cal M}
\def\D{\Delta}
\def \shat{{\hat{\sigma}}^2}
\def\uv{\mathbf{u}}
\def\l {\lambda}
\def\d{\delta}
\def\Sigmab{\boldsymbol{\Sigma}}
\def\Lambdab{\boldsymbol{\Lambda}}
\def\lambdab{\boldsymbol{\lambda}}
\def\Mg{{\cal M}_\gamma}
\def\S{{\cal{S}}}
\def\qg{p_{\boldsymbol{\gamma}}}
\def\pg{p_{\boldsymbol{\gamma}}}
\def\t{\boldsymbol{\theta}}  
\def\T{\boldsymbol{\Theta}}  
\begin{document}
{\bf STA721 \hfill Homework 1}

\vspace{.5in}
\noindent
Work the following problems from Christensen (C) and Wakefield (W)
\begin{enumerate}
\item 5.8 (W)  (see link to eBook on Calendar)
\item 1.5.8 (C) (see link to eBook on Calendar)

\item We showed that $\P_\X = \X(\X^T\X)^{-1}\X^T$ was an orthogonal
  projection on the column space of $\X$ and that $\hat{\Y} = P_\X \Y$.
  While useful for theory, the projection matrix should never be used
  in practice to find the MLE of $\mub$ due to 1) computational
  complexity (inverses and matrix multiplication) and instability.  To
  find $\hat{\b}$ we solve $\X \b = \P_\X \Y $ which
  leads to the {\it normal equations}  $(\X^T\X) \b = \X^T\Y$ and
  solving the system of equations for $\b$.
  Instead consider the following for $\X$ ($n \times p, p < n$) of rank $p$

  \begin{enumerate}
  \item Any $\X$ may be written via a singular value decomposition as
    $\U \Lambda \V^T$ where $\U$ is a $n \times p$ orthonormal matrix
    ($\U^T\U = \I_p$ and columns of $\U$ form an orthonormal basis (ONB) for
    $C(\X)$), $\Lambda$ is a $p \times p$ diagonal matrix and $\V$ is
    a $p \times p$ orthogonal matrix ($\V^T\V = \V \V^T = \I_p$. Note
    the difference between {\it orthonormal} and {\it orthogonal}.
    Show that $\P_X$ may be expressed as a function of $\U$ only and
    provide an expression for $\hat{\Y}$.  Similarly, find an
    expression for $hat{\b}$ in terms of $\U$, $\Lambda$ and $\V$.
    Your result should only require the inverse of a diagonal matrix!
\item $\X$ may be written in a (reduced or thinned) QR decomposition as a matrix
  $\Q$ that is a $n \times p$ orthonormal matrix (which forms an ONB
  for $C(\X)$) and $\mat{R}$ which is a $p
  \times p$ upper triangular matrix (i.e all elements below the
  diagonal are 0) where $\X = \Q \mat{R}$. The columns of $\Q$ are an ONB for
  the $C(\X)$. Show that $\P_\X$
  may be expressed as a function of $\Q$ alone.   Show that  the 
 the normal equations reduce to solving the triangular system $\mat{R} \b = \Z$ where $\Z = \Q^T \Y$.
 Because $\mat{R}$ is upper triangular, show that $\hat{\b}$ may be
 obtained be back-solving (and avoiding the matrix inverse of $\X^T\X$.
   
\item Any symmetric matrix $\A$ may be written via a Cholesky
  decomposition as $\A = \L \L^T$ where $\L$
  is lower triangular.   If $\Z = \X^T\Y$  show that we can solve two
  triangular systems $\L \L^T \b = \Z$ by solving for $\w$ using  $\L \w = \Z$ using a
  forward substitution and then for $\hat{\b}$ using $\L^T \b =
  \w$ avoiding any matrix inversion.

\item Use $\R$ to find $\Q$ and $\U$ for the matrices in problems 1.5.8 in
  Christensen. Does $\Q$ equal $\U$?   See help pages via {\tt
    help(qr)} and {\tt help(svd)} for function documentation.
\item Prove that the two projection matrices obtained by the SVD and
  the QR method are the same.  (Hint:  review Theorems in Christensen
  Appencicies about uniqueness of projections)

  \end{enumerate}
  Note: The Cholesky method is the fastest in terms of
  $O(n p^2 + p^3/3)$ floating point operations (flops), but is
  numerically unstable if the matrix is poorly conditioned.  R uses
  the QR method ($O(2 n p^2 - 2p^3/3)$ flops in the function {\tt lm.fit()}
  (which is the workhorse underneath the {\tt lm()} function.  Generalized QR
  algorithms can handle rank deficient case.  The SVD method is the
  most expensive $O(2 n p^2 + 11 n^3)$ but can handle the rank case.
  There are generalized Cholesky and QR methods for the rank deficient
  case.
\end{enumerate}
\end{document}

